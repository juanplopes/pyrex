% *****************************************************************************
% *****************************************************************************
% Guia para elaboração do arquivo fonte TEX para a geração de monografias,
%   dissertações e teses segundo as normas da UERJ.
% Autor: Luís Fernando de Oliveira (Prof. Dr. - DFAT/IF/UERJ)
% Versão: 0.1 - 08/05/2013
% *****************************************************************************
% *****************************************************************************
%
\documentclass[a4paper,12pt,oneside,onecolumn]{article}

\usepackage[brazil]{babel}  % adequacao para o portugues Brasil
\usepackage{cmap}           % Mapear caracteres especiais no PDF
\usepackage[utf8]{inputenc} % Determina a codificacao utiizada
                            % (conversão automática dos acentos)
\usepackage{makeidx}        % Cria o indice
\usepackage{hyperref}       % Controla a formacao do indice
\usepackage{lastpage}       % Usado pela Ficha catalografica
\usepackage{indentfirst}    % Indenta o primeiro paragrafo de cada secao.
\usepackage{color}          % Controle das cores
\usepackage{graphicx}       % Inclusao de graficos
\usepackage{amsmath} % pacote matemático
\usepackage[top=3cm, bottom=3cm, left=3cm, right=3cm]{geometry}
\usepackage{enumitem}        % 
\usepackage{caption}        % 

\newcommand\annexname{}
\usepackage{repUERJformat}        % 

\usepackage[alf]{abntex2cite}

% *****************************************************************************
% *****************************************************************************

\hyphenation{in-for-ma- re-co-men-da- a-fi-li-a-}

\newcommand{\comando}[1]{\texttt{\textbackslash #1}}
\newcommand{\opcoes}[1]{\texttt{[}\textsl{#1}\texttt{]}}
\newcommand{\param}[1]{\texttt{\{}\textsl{#1}\texttt{\}}}
\newcommand{\BibTeX}{{{Bib}}\TeX}
\newcommand{\repUERJ}{\textsf{repUERJ}}
\newcommand{\formato}[1]{\begin{flushleft}{#1}\end{flushleft}}

% *****************************************************************************
% *****************************************************************************
% Configurações de aparência do PDF final
% *****************************************************************************
% *****************************************************************************

% alterando o aspecto da cor azul
\definecolor{blue}{RGB}{41,5,195}

% informações do PDF
\hypersetup{
  %backref=true,
  %pagebackref=true,
  %bookmarks=true,                  % show bookmarks bar?
  pdftitle={\LaTeX\ e o pacote \repUERJ},%{\title{}},
  pdfauthor={Luís Fernando de Oliveira},%,{\author{}},
  pdfsubject={LaTeX com classe \repUERJ},
  pdfkeywords={PALAVRAS}{CHAVES}{chave1}{chave2}{chave3},
  pdfproducer={Luís Fernando de Oliveira},%,{\author{}}, % producer of the document
  pdfcreator={PDF\LaTeX},
  colorlinks=true,                  % false: boxed links; true: colored links
  linkcolor=blue,                   % color of internal links blue
  citecolor=red,                    % color of links to bibliography blue
  filecolor=magenta,                % color of file links magenta
  urlcolor=magenta,
  bookmarksdepth=4
}

% *****************************************************************************
% *****************************************************************************

% ---
\makeindex
% ---

\title{\LaTeX\ e o pacote \repUERJ}
\author{Luís Fernando de Oliveira
  \thanks{Dr., Prof. Adj. do Dep. de Física Aplicada e Termodinâmica (DFAT) do Instituto de Física (IF) da Universidade do Estado do Rio de Janeiro (UERJ). Documento versão 0.1 -- 08/05/2013.
  }
}

% *****************************************************************************
% *****************************************************************************

\begin{document}

% ---
\maketitle

\begin{abstract}
\noindent As normas da Universidade do Estado do Rio de Janeiro (UERJ) seguem as recomendações da Associação Brasileira de Normas Técnicas (ABNT). A confecção de documentos técnicos na UERJ é guiada por um documento disponibilizado pela Biblioteca Central da UERJ, base para a elaboração deste roteiro.
Apresentar-se-á, nos capítulos a seguir, as informações necessárias ao autor do documento para que o resultado final esteja em conformidade com as normas da UERJ.
\end{abstract}

\tableofcontents
% ---

%======================================================================================
\section*{Introdução}
%======================================================================================

As normas\index{normas} da Universidade do Estado do Rio de Janeiro (UERJ) seguem as recomendações da Associação Brasileira de Normas Técnicas (ABNT). A confecção de documentos técnicos na UERJ é guiada por um documento disponibilizado pela Biblioteca Central da UERJ, base para a elaboração deste roteiro.

Na Física, tradicionalmente, se utiliza o \LaTeX\ para a elaboração de relatórios e artigos. A elaboração de documentos técnicos usando o \LaTeX\ depende apenas da disponibilidade de um pacote apropriado que codifique as normas exigidas.

Neste sentido, tentou-se duas soluções iniciais: uma baseada no pacote desenvolvido pelo grupo chamado CódigoLivre.Org, coordenado por Gerald Weber, pacote este chamado de abn\TeX\footnote{Não há mais o link para o sítio do pacote. Acessdo em 03 de Abril de 2013.} e outra baseada na solução desenvolvida pelo Centro de Pesquisa em Arquitetura da Informação (CPAI) da Universidade de Brasília (UnB), chamado abn\TeX2\footnote{https://code.google.com/p/abntex2/}. Ambas as soluções codificam as normas da ABNT e, por isso, seriam uma base para o desenvolvimento de um pacote adequado a UERJ.

Problemas surgiram a partir do momento que o grupo de Gerald Weber deixou de manter as atualizações do pacote abn\TeX, isto em 2006. Mas a solução proposta pela equipe do CodigoLivre.Org foi bem vinda no sentido que as normas da ABNT foram implementadas a partir da classe \texttt{report.cls} permitindo um amplo acesso desde os comandos mais básicos até os mais elaborados.

Já a solução do CPAI se baseou na classe \texttt{memoir.cls} que é mais sofisticada que a classe \texttt{report.cls}, implementa uma série de comandos de alto nível para a configuração do \textsl{layout} do documento, comandos para manipulação dos estilos dos capítulos, seções, listas e outros, mas peca por ser pouco amigável na modificação da implementação para adequações mais específicas. Se, por um lado, a solução do CPAI resolveu alguns problemas da solução do CodigoLivre.Org, por outro, introduziu algumas dificuldades que não existiam na mesma.

Então, para resolver o impasse, decidiu-se criar um pacote próprio para as normas da UERJ, baseada na experiência das duas soluções anteriores: da abn\TeX, aproveitou-se a ideia de implementação a partir da classe \texttt{report.cls}; da abn\TeX2, aproveitou-se a ideia da construção de alguns comandos.

Assim sendo. o pacote criado implementa de forma mais direta e específica as normas da UERJ e se mantém acessível para futuras adequações caso as normas sejam alteradas.

Apresentar-se-á, nos capítulos a seguir, as informações necessárias ao autor do documento para que o resultado final esteja em conformidade com as normas da UERJ.

%======================================================================================
\section{Instalação do pacote \repUERJ}
%======================================================================================

A instalação do pacote \repUERJ\ (classe \texttt{repUERJ.cls} mais estilo \texttt{repUERJformat.sty}) se realiza a partir do \textsl{download} dos dois arquivos do pacote.

O autor pode usar tanto o Tex Live como o MikTex para processar os documentos. Ambos os aplicativos instalam o \LaTeX\ adequadamente em uma estrutura de diretórios que organizam, entre outras coisas, os diversos pacotes disponíveis para o processamento dos textos -- em especial, olhar o diretório \textsl{tmf/tex/latex/}. Dentro deste diretório, o autor deverá criar uma pasta chamada \textsl{UERJ} e copiar os dois arquivos do pacote \repUERJ\ para dentro dele.

A construção da bibliografia no estilo estabelecido pela ABNT, consequentemente pela, UERJ, é obtida através do pacote abn\TeX2. Este pacote está disponível para instalação automática tanto no Tex Live quanto no MikTex. Caso o autor não encontre o pacote pacote abn\TeX2 para instalação, deve-se proceder a instalação manual buscando o referido pacote na internet e seguindo as instruções próprias para instalação.

Para que o pacote \repUERJ\ seja reconhecido pelo \LaTeX, é necessário atualizar-se a lista de pacotes. No Tex Live, o autor deverá executar o \textsl{script texhash} ou \textsl{mktexlsr} com privilégios de administrador (superusuário). No MikTex, o autor deverá executar um aplicativo chamado \textsl{Settings} e optar por atualizar o banco de dados e pacotes (opções \textsl{Refresh FNDB} e \textsl{Update formats}). Isso é suficiente para que o pacote \repUERJ\ seja reconhecido durante o processamento do documento em elaboração.

Um cuidado deve ser tomado: não se deve deixar cópias de qualquer pacote espalhadas no computador, pois, ao atualizar o banco de dados e formatos, estas cópias serão registradas. E se por ventura um pacote for atualizado diferenciando-se das demais cópias, o \LaTeX\ poderá não gerar o resultado esperado por um simples motivo: durante o processamento, o \LaTeX\ pode utilizar uma cópia não atualizada gerando a sensação de que nada foi alterado. Este cuidado deve ser redobrado em relação ao pacote \repUERJ.

%======================================================================================
\section{Codificação inicial -- preâmbulo}
%======================================================================================

Todo documento gerado em \LaTeX\ inicia-se com um conjunto de comandos formando o que se chama de preâmbulo\index{preâmbulo}. O comando inicial é o que define a classe do documento:\\

\comando{documentclass}\opcoes{opções de configuração}\param{classe}\\

A classe desenvolvida para adequação das normas\index{normas} da UERJ se chama \texttt{repUERJ.cls}. Para um documento que use folha padrão A4, fonte tamanho 12, use somente a frente da folha e não possua divisão em colunas, o comando de classe será:\\

\comando{documentclass}\opcoes{a4paper,12pt,oneside,onecolumn}\param{repUERJ}\\

Depois da definição da classe, são inseridos os comandos para inclusão de pacotes específicos:\\

\comando{usepackage}\opcoes{opções de configuração}\param{pacote}\\

No caso deste documento, os pacotes são:\\

\comando{usepackage}\opcoes{brazil}\param{babel}  % adequacao para o portugues Brasil

\comando{usepackage}\param{cmap}                % Mapear caracteres especiais no PDF

\comando{usepackage}\opcoes{T1}\param{fontenc}    % Determina a codificação da fonte

\comando{usepackage}\opcoes{utf8}\param{inputenc} % Determina a codificação utiizada
                                             % (conversão automática dos acentos)

\comando{usepackage}\param{makeidx}        % Cria o indice

\comando{usepackage}\param{hyperref}       % Controla a formação do índice

\comando{usepackage}\param{lastpage}       % Usado pela Ficha catalográfica

\comando{usepackage}\param{indentfirst}    % Indenta o primeiro parágrafo de cada seção.

\comando{usepackage}\param{color}          % Controle das cores

\comando{usepackage}\param{graphicx}       % Inclusão de gráficos

\comando{usepackage}\opcoes{top=3cm, bottom=2cm, left=3cm, right=2cm}\param{geometry}\\

\comando{usepackage}\opcoes{alf}\param{abntex2cite}\\

\comando{usepackage}\opcoes{frame=no,gride=no,font=default}\param{repUERJformat}\\

Cada pacote atua de forma a introduzir, no processamento do documento, ações específicas, tais como:

\begin{itemize}
  \item pacote babel: modifica os títulos chaves de inglês para outra língua -- neste caso, português;
  \item pacote cmap: mapeia a tabela de caracteres a ser utilizado no formato pdf;
  \item pacote fontenc: determina a tabela de codificação da fonte;
  \item pacote inputenc: determina a tabela de codificação de caracteres utilizado no documento -- neste caso, UTF-8;
  \item pacote makeidx: permite a criação de um índice do documento;
  \item pacote hyperref: interfere na criação do índice introduzindo outras capacidades;
  \item pacote lastpage: sua função se resume a simplesmente retornar o número total de páginas do documento;
  \item pacote indentfirst: força que o primeiro parágrafo dos capítulos e seções sejam endentados;
  \item pacote color: introduz a capacidade de gerar textos coloridos;
  \item pacote graphicx: permite a manipulação de imagens;
  \item pacote geometry: permite a alteração do layout das páginas do documento -- neste caso, margem esquerda em 3cm, margem direita em 2cm, margem superior em 3cm e margem inferior em 2cm;
  \item pacote abntex2cite: permite a geração das referências bibliográficas segundo as normas da ABNT.\\
\end{itemize}

Algumas definições de estilo e formatação próprias do documento gerado a partir das normas da UERJ foram implementadas no pacote \texttt{repUERJformat}. Este pacote aceita três opções: exibição ou não da moldura da página (\opcoes{frame=yes \textbar no}), exibição ou não da grade de texto (\opcoes{gride=yes \textbar no}) e adoção da fonte \emph{Times}, \emph{Sans} ou o padrão do \LaTeX\ (\opcoes{font= times \textbar sans \textbar default}). Exemplo:\\

\begin{verbatim}
%-----------------------------------------------------------------------
% PREAMBULO
%-----------------------------------------------------------------------
\documentclass[a4paper,12pt,oneside,onecolumn]{repUERJ}

\usepackage[brazil]{babel}
\usepackage{cmap}
\usepackage[T1]{fontenc}
\usepackage[utf8]{inputenc}
\usepackage{makeidx}
\usepackage{hyperref}
\usepackage{lastpage}
\usepackage{indentfirst}
\usepackage{color}
\usepackage{graphicx}

\usepackage[top=3cm, bottom=2cm, left=3cm, right=2cm]{geometry}

\usepackage[frame=yes,gride=no,font=times]{repUERJformat}

\usepackage[alf]{abntex2cite}

%-----------------------------------------------------------------------
% FIM
%-----------------------------------------------------------------------

% Continuacao do documento com as informacoes institucionais, 
% de autoria e orientacao...
\end{verbatim}

%======================================================================================
\section{Informações institucionais e de autoria}
%======================================================================================

O documento de trabalho de final de curso é dividido em três conjuntos de elementos:

\begin{itemize}
  \item o conjunto dos elementos pré-textuais\index{elementos!pré-textuais}: capa\index{elementos!pré-textuais!capa}, folha de rosto\index{elementos!pré-textuais!folha de rosto}, folha de autorização\index{elementos!pré-textuais!folha de autorização} (no caso de monografia), folha de aprovação\index{elementos!pré-textuais!folha de aprovação} (no caso de dissertação e tese), dedicatória\index{elementos!pré-textuais!dedicatória}, agradecimentos\index{elementos!pré-textuais!agradecimentos}, epígrafe\index{elementos!pré-textuais!epígrafe}, resumos\index{elementos!pré-textuais!resumos}, listas e sumário\index{elementos!pré-textuais!listas}\index{elementos!pré-textuais!sumário});
  \item o conjunto de elementos textuais\index{elementos!textuais}: introdução\index{elementos!textuais!introdução}, desenvolvimento\index{elementos!textuais!desenvolvimento} e conclusão\index{elementos!textuais!conclusão};
  \item o conjunto de elementos pós-textuais\index{elementos!pós-textuais}: referências\index{elementos!pós-textuais!referências}, apêndices\index{elementos!pós-textuais!apêndices}, anexos\index{elementos!pós-textuais!anexos} e índice\index{elementos!pós-textuais!índice}.\\
\end{itemize}

Os elementos pré-textuais capa, folha de rosto, folha de autorização/aprovação e resumo em língua portuguesa utilizam informações iniciais que o autor deve fornecer antes dos comandos de impressão dos mesmos.

A capa apresenta o logo da UERJ, as informações institucionais, o nome do autor, o título do trabalho, o local e a data da apresentação. A folha de rosto, além destas informações, apresenta o nome do orientador e do coorientador (caso haja). A folha de autorização e de aprovação apresentam um texto adicional informando a natureza do trabalho impresso e outras informações, qual seja: se é monografia, dissertação ou tese; qual o grau a ser recebido; qual a área de concentração e a unidade/programa de pós-graduação onde o trabalho foi realizado. O resumo, por fim, traz, pelas normas\index{normas}, um parágrafo inicial que formata o referenciamento bibliográfico do documento.

Todas estas informações devem ser fornecidas pelo autor antes destes elemento serem gerados. Para isso, foram implementados os seguintes comandos:\\

\comando{autor}\param{nome do autor}\param{sobrenome do autor}

\comando{titulo}\param{título do trabalho}

\comando{orientador}\param{título}\param{nome do orientador}\param{sobrenome do orientador}\param{afiliação}

\comando{coorientador}\param{título}\param{nome do coorientador}\param{sobrenome do coorientador}\param{a\-fi\-li\-a\-ção}

\comando{grau}\param{grau a ser recebido}

\comando{curso}\param{nome do curso}

\comando{local}\param{local da apresentação}

\comando{data}\param{dia}\param{mês}\param{ano}\\

Os parâmetros \textsl{título} do orientador e coorientador permitem que o autor produza no texto impresso que precede os respectivos nomes as combinações, por exemplo: Prof.\ Dr.\ ou Profa.\ Dra.\ ou Prof.\ Adj.\ ou qualquer outro título que seja necessário ou conveniente. O comando \comando{grau} aceita como parâmetro as seguintes possibilidades:

\begin{itemize}
  \item \comando{grau}\param{Doutor}
  \item \comando{grau}\param{Mestre}
  \item \comando{grau}\param{Licenciado}
  \item \comando{grau}\param{Bacharel}
  \item \comando{grau}\param{Graduado}\\
\end{itemize}

O tipo de documento (monografia, dissertação ou tese) fica determinado a partir do grau e o autor não precisa se preocupar com esta informação.

Outro elemento importante que identifica a origem do trabalho é o logo. Pelas normas da UERJ, a capa deve apresentar o logo da instituição e a folha de rosto deve apresentar uma marca d'água. Estas duas imagens são identificadas pelo documento através dos comandos:\\

\comando{logo}\param{nome do arquivo de imagem}

\comando{marcadagua}\param{nome do arquivo de imagem}\param{escala}\param{coord. x}\param{coord. y}\\

Os arquivos de imagem devem estar no formato PNG ou JPG e de preferência no mesmo diretório que contém o arquivo TEX correspondente ao documento em elaboração. Se o autor desejar usar imagens com outras extensões, o mesmo deverá providenciar a inclusão dos pacotes necessários para esta necessidade. Os parâmetros \textsl{escala}, \textsl{coord. x} e \textsl{coord. y} permitem que o autor faça pequenos ajuste em relação a dimensão e posição\footnote{A origem do sistema de coordenadas é o canto inferior esquerdo da folha.} da marca d'água na folha de rosto. É importante que a marca d'água esteja centralizada vertical e horizontalmente.

Novamente, vejamos um exemplo destes comandos no contexto de um documento:\\

\begin{verbatim}
% Continuacao a partir do preambulo...

%-----------------------------------------------------------------------
% INFORMACOES FUNDAMENTAIS
%-----------------------------------------------------------------------

\logo{logo_uerj_cinza.png}
\marcadagua{marcadagua_uerj_cinza.png}{1}{160}{255}

\instituicao{Universidade do Estado do Rio de Janeiro}
            {Centro de Tecnologia e Ciência}
            {Instituto de Física}
            {Armando Dias Tavares}

\autor{Luís Fernando de}{Oliveira}
\titulo{\LaTeX\ e as normas da UERJ para elaboração de trabalhos técnicos}

\orientador{Prof.\ Dr.}
           {João Manuel de}{Abreu}
           {Universidade do Estado do Rio de Janeiro}

\coorientador{Profa.\ PhD.}
             {Zulmira Souza}{Franco}
             {Universidade Vila Princesa Isabel}

\grau{Doutor}
\curso{Física}

\local{Rio de Janeiro}
\data{25}{Janeiro}{2013}

%-----------------------------------------------------------------------
% FIM
%-----------------------------------------------------------------------

% continuacao do documento com os elementos pre-textuais...
\end{verbatim}

%======================================================================================
\section{Elementos pré-textuais}
%======================================================================================

A tabela \ref{tab:pretextuais} apresenta os elementos pré-textuais\index{elementos!pré-textuais} segundo as normas:

\begin{table}[h!]
	\centering
	\caption{Elementos pré-textuais}\label{tab:pretextuais}
	\begin{tabular}{l|l}%|
		\hline
		Elemento & Inserção\\
		\hline
		Capa & obrigatório\\
		Folha de rosto & obrigatório\\
		Folha de autorização& obrigatório\\
		Folha de aprovação & obrigatório\\
		Dedicatória & opcional\\
		Agradecimentos & opcional\\
		Epígrafe & opcional\\
		Resumo em portugues & obrigatório\\
		Resumo em língua estrangeira & obrigatório\\
		Lista de ilustrações & opcional\\
		Lista de tabelas& opcional\\
		Lista de abreviaturas e siglas & opcional\\
		Lista de símbolos & opcional\\
		Sumário & obrigatório\\
		\hline
	\end{tabular}
\end{table}

A capa e a folha de rosto são geradas pelos comandos \comando{capa} e \comando{folhaderosto}, respectivamente.

A folha de autorização, para o caso da monografia, não depende de nenhuma outra informação adicional. Logo, é gerada pelo comando \comando{folhadeautorizacao}.

Já a folha de aprovação depende da inserção dos nomes dos membros da banca. Como esta informação não tem como ser prevista (por conta do número de membros, que pode flutuar), optou-se por criar-se um ambiente chamado \textsl{folhadeaprovacao} para que a relação de nomes dos membros da banca examinadora pudesse ser inserida. O comando é:\\

\comando{begin}\param{folhadeaprovacao}

\comando{assinatura}\param{nome do membro 1\textbackslash\textbackslash afiliação 1}

\comando{assinatura}\param{nome do membro 2\textbackslash\textbackslash afiliação 2}

.........

\comando{end}\param{folhadeaprovacao}\\

O autor deve inserir tantos comandos \comando{assinatura} quando forem necessários para registrar a relação completa de membros da banca. É importante destacar que não é necessário a inclusão do orientador (e coorientador) na relação, pois isso é feito automaticamente pelo ambiente \textsl{folhadeaprovacao}.

A partir da folha de autorização/aprovação, os elementos pré-textuais não possuem uma diagramação específica, seguem o modelo de uma página textual com um título centralizado seguido do texto pertinente ao elemento. É o caso da dedicatória, agradecimentos, epígrafe e resumos. Na preparação da classe \texttt{repUERJ.cls}, optou-se pela implementação de um comando específico para estes elementos. O comando é:\\

\comando{pretextualchapter}\param{título do elemento}\\

Para a geração da página de dedicatória, o autor deve inserir o comando:\\

\comando{pretextualchapter}\param{Dedicatória}\\

\noindent seguido do texto que exprime a dedicatória.

Para a geração da página de dedicatória, o autor deve inserir o comando:\\

\comando{pretextualchapter}\param{Agradecimentos}\\

\noindent seguido do texto que exprime os agradecimentos.

Para a geração da página de epígrafe, o autor deve inserir o comando:\\

\comando{pretextualchapter}\param{}\\

\noindent seguido do pensamento e do autor do mesmo.

O resumo em língua portuguesa também deve ser inserido através dos comandos:\\

\comando{pretextualchapter}\param{Resumo}

\comando{refbibliografica}\\

\noindent O comando \comando{refbibliografica} gera a estrutura de referência bibliográfica do documento que está sendo elaborado. Após este comando, o autor deve inserir o resumo propriamente dito. No final, ele deve destacar também as palavras chaves. Não um comando específico para isso.

O resumo em língua estrangeira tem a mesma estrutura dos demais elementos citados. O comando é:\\

\comando{pretextualchapter}\param{Abstract}\\

\noindent se resumo for em inglês ou\\

\comando{pretextualchapter}\param{Resum\`e}\\

\noindent se for em francês, seguido do texto do resumo traduzido na língua estrangeira e das palavras-chaves.

Na sequência, tem-se a geração das listas de ilustrações e tabelas. Os comandos para a inclusão destes elementos no documento são:\\

\comando{listadefiguras}

\comando{listadetabelas}\\

O guia prevê outras listas opcionais: lista de algoritmos, lista de abreviaturas e lista de símbolos. Os comandos são:\\

\comando{listadealgoritmos}\\

\comando{pretextualchapter}\param{Lista de abreviaturas e siglas}

\comando{abreviatura}\param{ABREVIATURA1}\param{Texto 1}

\comando{abreviatura}\param{ABREVIATURA2}\param{Texto 2}

\comando{abreviatura}\param{ABREVIATURA3}\param{Texto 3}\\

\comando{pretextualchapter}\param{Lista de símbolos}

\comando{simbolo}\param{simbolo1}\param{definição 1}

\comando{simbolo}\param{simbolo2}\param{definição 2}

\comando{simbolo}\param{simbolo3}\param{definição 3}\\

Por fim, para a geração do sumário:\\

\comando{sumario}\\

Exemplo da aplicação destes comandos na elaboração do documento:\\

\begin{verbatim}
% Continuacao a partir das informacoes fudamentais...

\begin{document}

% ----------------------------------------------------------
% ELEMENTOS PRE-TEXTUAIS
% ----------------------------------------------------------

\frontmatter

% ----------------------------------------------------------
% Capa e a folha de rosto
% ----------------------------------------------------------

\capa
\folhaderosto

% ----------------------------------------------------------
% Folha de aprovação
% ----------------------------------------------------------

\begin{folhadeaprovacao}
  \assinatura{Prof.\ Dr.\ Jorge Hudge\\ Centro de Pesquisa Itatiaia}
  \assinatura{Prof.\ Dr.\ Silva Pinto\\ Faculdade de Itaboraí}
  \assinatura{Prof.\ Dr.\ Francisco Bicalho\\ Universidade Rio das Ostras}
\end{folhadeaprovacao}

% ----------------------------------------------------------
% Dedicatoria
% ----------------------------------------------------------

\pretextualchapter{Dedicatória}

  \vfill\vfill
  \begin{center}
  \begin{minipage}{.8\textwidth}
    Aos pesquisadores.
  \end{minipage}
  \end{center}
  \vfill

% ----------------------------------------------------------
% Agradecimentos
% ----------------------------------------------------------

\pretextualchapter{Agradecimentos}

A todos que nos auxiliaram nessa jornada, nosso muito obrigado.

% ----------------------------------------------------------
% Epigrafe
% ----------------------------------------------------------

\pretextualchapter{}

  \vfill\vfill\vfill\vfill
  \begin{flushright}
    \textsl{A verdade está lá fora.\\ Fox Molder}
  \end{flushright}
  \vfill

% ----------------------------------------------------------
% Resumo em portugues
% ----------------------------------------------------------

\pretextualchapter{Resumo}

\refbibliografica

Este é o resumo em portugues.\\

\noindent {Palavras-chave}: Chave 1. Chave 2. Chave 3.

% ----------------------------------------------------------
% Abstract
% ----------------------------------------------------------

\pretextualchapter{Abstract}

This is the english abstract.\\

\noindent {Keywords}: Word 1. Word 2. Word 3.

% ----------------------------------------------------------
% Listas de ilustracoes e tabelas
% ----------------------------------------------------------

\listadefiguras
\listadetabelas

% ----------------------------------------------------------
% Lista de abreviaturas e siglas
% ----------------------------------------------------------

\pretextualchapter{Lista de abreviaturas e siglas}

\abreviatura{DF}{Distrito Federal}
\abreviatura{ONU}{Organização das Nações Unidas}
\abreviatura{UERJ}{Universidade do Estado do Rio de Janeiro}

% ----------------------------------------------------------
% Lista de simbolos
% ----------------------------------------------------------

\pretextualchapter{Lista de símbolos}

\simbolo{\alpha}{aceleração angular}
\simbolo{\sigma^2}{variância}
\simbolo{\mu}{coeficiente de atenuação linear}

% ----------------------------------------------------------
% Sumario
% ----------------------------------------------------------

\sumario

% ----------------------------------------------------------
% FIM
% ----------------------------------------------------------

% continuacao do documento com os elementos textuais...
\end{verbatim}

%======================================================================================
\section{Elementos Textuais}
%======================================================================================

Os elementos textuais\index{elementos!textuais} englobam: introdução, desenvolvimento e conclusão.

A introdução é um capítulo obrigatório na elaboração do documento. O desenvolvimento incorpora todos os demais capítulos que constituem o trabalho. A conclusão também é um elemento obrigatório e fecha o corpo principal do documento.

A introdução é o único elemento do conjunto textual que não é numerado. O comando para inserir a introdução é:\\

\comando{chapter*}\param{Introdução}\\

\noindent lembrando que o asterisco impede que o capítulo seja numerado. Todos os demais capítulos numerados utilizam o comando padrão:\\

\comando{chapter}\param{título do capítulo}\\

A hierarquia de seções (seção, subseção, subsubseção) é implementada usando os comandos tradicionais do \LaTeX:\\

\comando{section}\param{título da seção}\\

\comando{subsection}\param{título da subseção}\\

\comando{subsubsection}\param{título da subsubseção}\\

O modelo \repUERJ\ foi desenvolvido de forma a refletir os estilos estabelecidos pelas normas da UERJ para capítulos, seções, subseções e subsubseções, isto é, caixa alta e negrito para capítulos, negrito para seções, sublinhado para subseções e estilo padrão de texto para as subsubseções. O autor não precisa se preocupar com isto.

Além destes comados de estruturação do documento, o autor tem todos os demais comandos do \LaTeX\ para usar: inclusão de figuras, tabelas, enumerações, alíneas, comandos para justificar a direita, esquerda, alteração de espaçamento, inclusão de citações longas, etc.

Por fim, o \LaTeX\ usa tradicionalmente os títulos destes elementos para compor o sumário que é gerado, a partir do modelo \repUERJ com a formatação correta segundo as normas (idem para as figuras e tabelas e demais listas que se deseje criar).

Veja um exemplo de estrutura textual:\\

\begin{verbatim}
% ----------------------------------------------------------
% ELEMENTOS TEXTUAIS
% ----------------------------------------------------------

\mainmatter

% ----------------------------------------------------------
\chapter*{Introdução}
% ----------------------------------------------------------

Texto da introdução.

% ----------------------------------------------------------
\chapter{Fundamentos teóricos}
% ----------------------------------------------------------

Texto da fundamentação teórica...

% ----------------------------------------------------------
\section{Efeito O'bladi}
% ----------------------------------------------------------

Texto da seção...

% ----------------------------------------------------------
\section{Efeito O'blada}
% ----------------------------------------------------------

Texto da seção...

% ----------------------------------------------------------
\chapter{Fundamentos metodológicos}
% ----------------------------------------------------------

Texto da metodologia...

% ----------------------------------------------------------
\section{Materiais}
% ----------------------------------------------------------

Texto da seção...

% ----------------------------------------------------------
\section{Métodos}
% ----------------------------------------------------------

Texto da seção...

% ----------------------------------------------------------
\chapter{Resultados}
% ----------------------------------------------------------

Texto dos resultados...

% ----------------------------------------------------------
\chapter{Conclusão}
% ----------------------------------------------------------

Texto da conclusão...

% ----------------------------------------------------------
% FIM
% ----------------------------------------------------------
\end{verbatim}

%======================================================================================
\section{Elementos pós-textuais}
%======================================================================================

Os elementos pós-textuais\index{elementos!pós-textuais} incluem: referências bibliográficas, apêndices, anexos e índice.

A construção das referências bibliográficas será explorada na seção \ref{cap:referencias}. Mas é importante apresentar aqui que o autor pode usar ou o ambiente \texttt{thebibliography} ou outro comando provido para pacotes próprios para gerenciamento de bibliografias, por exemplo o \BibTeX.

A parte referente aos apêndices deve ser iniciada com o comando:\\

\comando{appendix}\\

\noindent Cada apêndice é previsto como se fosse um capítulo. Como o estilo de impressão dos apêndices e anexos é diferente do estilo de um capítulo do corpo textual, optou-se por se criar um comando próprio chamado:\\

\comando{postextualchapter}\param{título do elemento pós-textual}\\

\noindent Isso gerará uma entrada no sumário e a impressão do apêndice (também válido para anexo) com o estilo correto. A partir daí, o autor cria a estrutura textual dentro do apêndice (ou anexo) da mesma forma que ele faria com um capítulo.

Assim como o conjunto de apêndices é iniciado com o comando \comando{appendix}, o conjunto de anexos é iniciado com o comando \comando{annex}, os títulos dos anexos são inseridos pelo comando \comando{postextualchapter}, como no apêndice, e a estrutura textual segue a forma padrão de elaboração. Nada diferente do que foi escrito para o tratamento dos apêndices.

Segue um exemplo de estrutura pós-texutal:\\

\begin{verbatim}
% ----------------------------------------------------------
% ELEMENTOS PÓS-TEXTUAIS
% ----------------------------------------------------------

\backmatter

% ----------------------------------------------------------
% Referências bibliográficas
% ----------------------------------------------------------

\citeoption{abnt-options4}
\bibliography{abnt-options4,bibliografia,modelo_bibtex}

% ----------------------------------------------------------
% Apendices
% ----------------------------------------------------------

\appendix

% ----------------------------------------------------------
\postextualchapter{Primeiro apêndice}
% ----------------------------------------------------------

\section{Primeira seção}

\subsection{Primeira subseção}

\subsubsection{Primeira subsubseção}

\subsubsection{Segunda subsubseção}

\subsection{Segunda subseção}

\section{Segunda seção}

% ----------------------------------------------------------
\postextualchapter{Segundo apêndice}
% ----------------------------------------------------------

\section{Primeira seção}

\subsection{Primeira subseção}

\subsection{Segunda subseção}

\subsubsection{Primeira subsubseção}

\subsubsection{Segunda subsubseção}

\section{Segunda seção}

% ----------------------------------------------------------
% Anexos
% ----------------------------------------------------------

\annex

% ----------------------------------------------------------
\postextualchapter{Primeiro anexo}
% ----------------------------------------------------------

\section{Primeira seção}

\subsection{Primeira subseção}

\subsubsection{Primeira subsubseção}

\subsubsection{Segunda subsubseção}

\subsection{Segunda subseção}

\section{Segunda seção}

% ----------------------------------------------------------
\postextualchapter{Segundo anexo}
% ----------------------------------------------------------

\section{Primeira seção}

\subsection{Primeira subseção}

\subsubsection{Primeira subsubseção}

\subsubsection{Segunda subsubseção}

\subsection{Segunda subseção}

\section{Segunda seção}

% ----------------------------------------------------------
% FIM
% ----------------------------------------------------------
\end{verbatim}

%======================================================================================
\section{Inserindo figuras e tabelas}
%======================================================================================

O \LaTeX\ permite a inserção de figuras e tabelas como objeto flutuante ou não. Pelas normas, toda figura e tabela deve ser acompanhada de um texto explicativo, rótulo (\textsl{caption}). Este texto é utilizado também na criação da lista de ilustrações e de tabelas.

Logo, é necessário um pouco de cuidado na hora de preencher este rótulo. Primeiro, bom senso no sentido de criar-se um texto que contextualize a informação não textual. Não deve ser um texto longo. O local próprio para isto é no corpo de texto que deve sempre referenciar o elemento de informação não textual. Segundo, não se insere no rótulo a origem da informação. Para isto existe a legenda (\textsl{legend}) que não é numerada e não participa das listas no início do documento, mas que deve aparecer logo abaixo do rótulo. Se a ilustração ou tabela é criação inédita, a legenda é dispensada.

O comando para inserção de figuras no \LaTeX\ é:\\

\comando{includegraphics}\opcoes{opções}\param{caminho e nome da figura}\\

Este comando não cria um objeto flutuante e não permite a inserção de rótulos e legendas. Para isso, é necessário usar o ambiente \textsl{figure}. Os comando para inserção do rótulo e da legenda são:\\

\comando{caption}\param{texto explicativo}\\

\comando{legend}\param{palavra-chave da legenda}\param{texto explicativo}\\

Seguem um exemplo:

\begin{verbatim}
\begin{figure}[ht]
  \centering
  \includegraphics[scale=0.50]{logo_uerj_cor.jpg}
  \caption{Logomarca da UERJ.} \label{fig:uerj1}
  \legend{Fonte}{Normas de aplicação logomarcas -- 
                 http://www.uerj.br/institucional/.}
\end{figure}
\end{verbatim}

\begin{figure}[ht]
  \centering
  \includegraphics[scale=0.50]{logo_uerj_cor.jpg}
  \caption{Logomarca da UERJ.} \label{fig:uerj1}
  \caption*{Fonte: Normas de aplicação logomarcas -- http://http://www.uerj.br/institucional/.}
\end{figure}

Para as tabelas, a construção das mesmas é realizada pelos ambientes \textsl{tabular}, \textsl{tabbing}, e outros similares. Mas o objeto flutuante é construído pelo ambiente \textsl{table}. Este último permite, similar ao ambiente \textsl{figure}, a inserção de rótulo e legenda. Segue um exemplo:

\begin{verbatim}
\begin{table}[ht]
  \centering
  \caption{Tabela de valores.}
  \begin{tabular}{l|l}
    \hline
      X & Y\\
    \hline
      1,20 & 15,7\\
      1,23 & 15,6\\
      1,19 & 15,3\\
      1,26 & 15,1\\
      1,22 & 15,5\\
      1,16 & 15,3\\
      1,37 & 15,7\\
    \hline
  \end{tabular}
\end{table}
\end{verbatim}

\begin{table}[ht]
  \centering
  \caption{Tabela de valores.}
  \begin{tabular}{l|l}
    \hline
      X & Y\\
    \hline
      1,20 & 15,7\\
      1,23 & 15,6\\
      1,19 & 15,3\\
      1,26 & 15,1\\
      1,22 & 15,5\\
      1,16 & 15,3\\
      1,37 & 15,7\\
    \hline
  \end{tabular}
\end{table}

%======================================================================================
\section{Inserindo algoritmos}
%======================================================================================

Para se garantir compatibilidade do estilo de escrita do documento e da inserção de algoritmos, optou-se por se criar um ambiente próprio para os algoritmos. Existem vários pacotes destinados à escrita de algoritmos no \LaTeX. Porém, cada um traz características que ou não são interessantes ou conflitam com o pacote \repUERJ.

Dos vários pacotes disponíveis para \LaTeX\ (\texttt{alg}, \texttt{algorithm2e}, \texttt{algorithmicx} e \texttt{algorithms} entre outros), algumas características foram incorporadas no ambiente \texttt{algorithm} do pacote \repUERJ. Uma delas é a quebra de algoritmos muito longos. Outra característica é a inserção de comandos próprios para a criação da documentação completa do algoritmo, incluindo: propósito, método, entradas, saídas, observações e comentários.

O ambiente \textsl{algorithm} define o espaço para a digitação do algoritmo. Está implementado como um objeto flutuante. Logo aceita qualquer combinação das opções de posicionamento \opcoes{!htbp}. Além disso, o autor pode inserir um rótulo (\comando{caption}) para o algoritmo que será numerado continuamente e inserido na lista de algoritmos.

Para implementar a quebra de algoritmos, criou-se um outro ambiente \textsl{algorithm*} que imprime a mesma numeração da parte inicial do algoritmo já escrito, ou seja, este novo ambiente garante tudo que o ambiente \textsl{algorithm} prevê, porém não incrementa o contador do rótulo.

Adicionalmente, criou-se um outro ambiente para a inserção propriamente dita da documentação completa do algoritmo. Este ambiente se chama \textsl{pseudocode}. Ele controla as endentações do algoritmo automaticamente e a numeração das linhas de instrução. Por uma questão de coerência com a quebra de algoritmos com o ambiente \textsl{algorithm*}, criou-se também o ambiente \textsl{pseudocode*} que não `zera' a contagem das linhas de instrução gerando o efeito de contagem contínua entre as partes quebradas.

Como facilidade para o autor, a numeração das linhas do algoritmo podem ser `ligadas' e `desligadas' a qualquer momento através dos comandos:\\

  \comando{alglinenumberson}, para `ligar' a numeração das linhas,\\

  \comando{alglinenumbersoff}, para `desligar' a numeração das linhas\\

Estes comandos devem ser usados antes dos ambientes \textsl{algorithm} e \textsl{algorithm*} e podem ser alternados sempre que o autor desejar.

Descreveremos a seguir os comandos que implementam estas características chamadas de \textsl{estrutura básica de documentação de algoritmo}.

\begin{itemize}
  \item Início e fim da documentação completa:\\
    \comando{Documentacao}\\
    \comando{FimDocumentacao}
  \item Título:\\
    \comando{Titulo}\param{título da documentação}
  \item Propósito:\\
    \comando{Proposito}\param{texto explicativo}
  \item Método:\\
    \comando{Metodo}\param{texto explicativo}
  \item Entradas:\\
    \comando{Entradas}\param{texto explicativo}
  \item Saídas:\\
    \comando{Saidas}\param{texto explicativo}
  \item Observações:\\
    \comando{Observacoes}\param{texto explicativo}
  \item Início e fim do algoritmo:\\
    \comando{Algoritmo}\param{nome do algoritmo (opcional)}\\
    \comando{FimAlgoritmo}
\end{itemize}

As instruções características do algoritmos sâo:

\begin{itemize}
  \item Declaração de variáveis e constantes:\\
    \comando{Declarar}\param{lista de variáveis|constantes}\param{tipo}\param{inicializações}
  \item Instrução de entrada:\\
    \comando{Ler}\param{lista de variáveis}
  \item Instrução de saída:\\
    \comando{Escrever}\param{lista de \textsl{strings}, variáveis e expressões}
  \item Instrução de desvio:\\
    \comando{SeEntao}\opcoes{comentário}\param{teste lógico}\\
    \comando{SenaoSeEntao}\opcoes{comentário}\param{teste lógico}\\
    \comando{Senao}\opcoes{comentário}\\
    \comando{FimSe}
  \item Instruções de repetição:\\
    \comando{Enquanto}\opcoes{comentário}\param{teste lógico}\\
    \comando{FimEnquanto}\\

    \comando{Fazer}\opcoes{comentário}\\
    \comando{Enquanto}\param{teste lógico}\\

    \comando{Repetir}\opcoes{comentário}\\
    \comando{AteQue}\param{teste lógico}\\

    \comando{ParaDeAte}\opcoes{comentário}\param{contador}\param{valor inicial}\param{valor final}\param{passo}\\
    \comando{FimPara}\param{texto explicativo}
  \item Interrupção de repetição:\\
    \comando{Parar}
  \item Retorno de rotina:\\
    \comando{Retornar}\param{variável}
  \item Início e fim do algoritmo:\\
    \comando{Algoritmo}\\
    \comando{FimAlgoritmo}
\end{itemize}

Além destes comandos, temos ainda:

\begin{itemize}
  \item Comentário:\\
    \comando{Comentario}\param{texto}
  \item Linha em branco:\\
    \comando{LinhaEmBranco}
  \item Construção de \textsl{strings}:\\
    \comando{String}\param{texto}
\end{itemize}


\begin{verbatim}
\begin{algorithm}[!ht]
    \caption{Congruência multiplicativa.} \label{alg:congmult}
    \begin{pseudocode}
      \LinhaEmBranco
      \Documentacao
        \Titulo{Congruência multiplicativa\\}
        \Proposito{Nenhum.\\}
        \Metodo{Nenhum.\\}
        \Entradas{
          a, m: multiplicador e módulo\\
          n0: semente\\
          i: contador auxiliar\\
        }
        \Saidas{
          n: número aleatório\\
        }
        \Observacoes{Nenhuma.\\}
        \Algoritmo{CMult}
          \Declarar{a, m, i}{numéricos}{}
          \Declarar{n0, n}{numéricos}{}
          \Ins{$m \leftarrow 13$}
          \Ins{$n0 \leftarrow 1$}
          \ParaDeAte[para cada possível valor de `a']{$a$}{2}{$m-1$}{}
            \Escrever{``a = '', a, ``: n = \{''}
            \Ins[reinicia a geração com a semente n0]{$n \leftarrow n0$}
            \Continua
    \end{pseudocode}
\end{algorithm}
\end{verbatim}

\begin{algorithm}[!ht]
    \caption{Congruência multiplicativa.} \label{alg:congmult}
    \begin{pseudocode}
      \LinhaEmBranco
      \Documentacao
        \Titulo{Congruência multiplicativa\\}
        \Proposito{Nenhum.\\}
        \Metodo{Nenhum.\\}
        \Entradas{
          a, m: multiplicador e módulo\\
          n0: semente\\
          i: contador auxiliar\\
        }
        \Saidas{
          n: número aleatório\\
        }
        \Observacoes{Nenhuma.\\}
        \Algoritmo{CMult}
          \Declarar{$a, m, i$}{numéricos}{}
          \Declarar{$n0, n$}{numéricos}{}
          \Ins{$m \leftarrow 13$}
          \Ins{$n0 \leftarrow 1$}
          \ParaDeAte[para cada possível valor de `a']{$a$}{2}{$m-1$}{}
            \Escrever{``a = '', $a$, ``: n = \{''}
            \Ins[reinicia a geração com a semente n0]{$n \leftarrow n0$}
            \Continua
    \end{pseudocode}
\end{algorithm}

\begin{verbatim}
\alglinenumbersoff
\begin{algorithm*}[!ht]
    \caption{Congruência multiplicativa.}
    \begin{pseudocode*}
            \LinhaEmBranco
            \Continuacao
            \ParaDeAte{$i$}{0}{$m-1$}{}
              \Ins[gerador de números aleatórios]{$n \leftarrow resto(a*n, m)$}
              \SeEntao[se fim da sequencia ...]{$n == n0$}
                \Escrever{$n$,``\}''}
                \Parar
              \Senao
                \Escrever{$n$}
              \FimSe
            \FimPara
          \FimPara
        \FimAlgoritmo
      \FimDocumentacao
    \end{pseudocode*}
\end{algorithm*}
\end{verbatim}

\alglinenumbersoff
\begin{algorithm*}[!ht]
    \caption{Congruência multiplicativa.}
    \begin{pseudocode*}
            \LinhaEmBranco
            \Continuacao
            \ParaDeAte{$i$}{0}{$m-1$}{}
              \Ins[gerador de números aleatórios]{$n \leftarrow resto(a*n, m)$}
              \SeEntao[se fim da sequencia ...]{$n == n0$}
                \Escrever{$n$,``\}''}
                \Parar
              \Senao
                \Escrever{$n$}
              \FimSe
            \FimPara
          \FimPara
          \LinhaEmBranco
          \Ins{$a \leftarrow 1$}
          \Enquanto[comentário]{$a<10$}
            \Escrever{$a$}
            \Ins{$a \leftarrow a+1$}
          \FimEnquanto
          \LinhaEmBranco
          \Ins{$a \leftarrow 1$}
          \Repetir[comentário]
            \Escrever{$a$}
            \Ins{$a \leftarrow a+1$}
          \AteQue{$a\ge10$}
          \LinhaEmBranco
          \Ins{$a \leftarrow 1$}
          \Fazer[comentário]
            \Escrever{$a$}
            \Ins{$a \leftarrow a+1$}
          \Enquanto{$a<10$}
        \FimAlgoritmo
      \FimDocumentacao
    \end{pseudocode*}
\end{algorithm*}


\begin{verbatim}
\alglinenumberson
\begin{algorithm}[!ht]
    \caption{Gerador de número aleatório usando congruência multiplicativa.}
    \label{alg:aleatorio}
    \begin{pseudocode}
      \LinhaEmBranco
      \Documentacao
        \Titulo{Função Aleatório\\}
        \Proposito{Nenhum.\\}
        \Metodo{Nenhum.\\}
        \Entradas{
          a, m: multiplicador e módulo\\
          n0: semente\\
        }
        \Saidas{
          n: número aleatório\\
        }
        \Observacoes{Nenhuma.\\}
        \Funcao{Aleatorio}{$a,n0,m$}
          \Declarar{$a, n0, m$}{numéricos}{}
          \Declarar{$n$}{numérico}{}
          \Ins[gerador de números aleatórios]{$n \leftarrow resto(a*n0, m)$}
          \Retornar{$n$}
        \FimFuncao
      \FimDocumentacao
    \end{pseudocode}
\end{algorithm}
\end{verbatim}

\alglinenumberson
\begin{algorithm}[!ht]
    \caption{Gerador de número aleatório usando congruência multiplicativa.}
    \label{alg:aleatorio}
    \begin{pseudocode}
      \LinhaEmBranco
      \Documentacao
        \Titulo{Função Aleatório\\}
        \Proposito{Nenhum.\\}
        \Metodo{Nenhum.\\}
        \Entradas{
          a, m: multiplicador e módulo\\
          n0: semente\\
        }
        \Saidas{
          n: número aleatório\\
        }
        \Observacoes{Nenhuma.\\}
        \Funcao{Aleatorio}{$a,n0,m$}
          \Declarar{$a, n0, m$}{numéricos}{}
          \Declarar{$n$}{numérico}{}
          \Ins[gerador de números aleatórios]{$n \leftarrow resto(a*n0, m)$}
          \Retornar{$n$}
        \FimFuncao
      \FimDocumentacao
    \end{pseudocode}
\end{algorithm}

%======================================================================================
\section{Escrevendo referências e bibliografia}\label{cap:referencias}
%======================================================================================

Certamente, em todo trabalho técnico-acadêmico, a etapa mais enfadonha é a construção, segundo as normas técnicas, das referências e da bibliografia (indicadas, a partir daqui, apenas como ``referências'').

No \LaTeX\ existem diferentes formas de se compor as referências. A mais simples (no sentido de recursos), que não depende de nenhum pacote específico, é a construção da bibliografia usando o ambiente \texttt{thebiblio\-graphy}. Este método gera uma lista de referências bibliográficas numeradas.

Outro caminho possível é o uso da ferramenta \BibTeX\ que gera uma lista alfabética de entradas (estilo autor-data) e organiza a base de dados bibliográficos através de um arquivo com formato próprio. A base de dados em si é uma coleção de entradas compostas por campos. Diferentes tipos de entrada estão disponíveis para suportar as categorias mais comuns de bibliografia, tais como artigo, relatório técnico, \textsl{proceedings}, manual, livro e tese. 

Cada categoria possui uma lista de entradas obrigatórias e opcionais. Cada entrada é formatada de forma diferente em função da categoria. Então, o esforço na aplicação do \BibTeX\ reside no reconhecimento de que categoria e que entradas devem ser preenchidas para que a saída (o efeito final de formatação) seja compatível com o estilo da referência esperada. Porém, uma vez detectada a combinação certa, a prática recai na reutilização da combinação para os outros casos.

Outro fato é que o \BibTeX\ possui uma coleção de categorias que não corresponde biunivocamente com as categorias relacionadas pelas normas da UERJ. Por isso, a busca pela combinação certa de entradas, dentro de uma relação de possíveis categorias que trará o efeito de formatação correto, é um passo importante apesar de cansativo. Por isso, o esforço neste capítulo será o de apresentar as categorias identificadas pelas normas da UERJ e identificar a categoria dentro do \BibTeX\ e a combinação de entradas que gerará a formatação correta.

%::::::::::::::::::::::::::::::::::::::::::::::::::::::::::::::::::::::::::::
\subsection{Entradas definidas no \BibTeX}
%::::::::::::::::::::::::::::::::::::::::::::::::::::::::::::::::::::::::::::

\begin{itemize}
  \item Para artigos em periódicos: 

  \formato{\citetext{bib:article}}

  \formato{
    @article\{bib:article,\\
      author = \{author article\},\\
      title = \{title article\},\\
      journal = \{journal\},\\
      publisher = \{publisher\},\\
      address = \{address\},\\
      volume = \{volume\},\\
      number = \{number\},\\
      pages = \{pages\},\\
      month = \{month\},\\
      year = \{year\},\\
      note = \{note\},\\
      url = \{url\},\\
    \}
  }

  \item Para artigos em periódicos sem o campo author: 

  \formato{\citetext{bib:article1}}

  \formato{
    @article\{bib:article1,\\
      organization = \{organization article\},\\
      title = \{title article\},\\
      journal = \{journal\},\\
      publisher = \{publisher\},\\
      address = \{address\},\\
      volume = \{volume\},\\
      number = \{number\},\\
      pages = \{pages\},\\
      month = \{month\},\\
      year = \{year\},\\
      note = \{note\},\\
      url = \{url\},\\
    \}
  }
\end{itemize}

\begin{itemize}
  \item Para livretos: 

  \formato{\citetext{bib:booklet}}

  \formato{
    @book\{bib:booklet,\\
      author = \{author booklet\},\\
      type = \{type\},\\
      title = \{title booklet\},\\
      edition = \{edition\},\\
      address = \{address\},\\
      volume = \{volume\},\\
      number = \{number\},\\
      series = \{series\},\\
      month = \{month\},\\
      year = \{year\},\\
      pages = \{pages\},\\
      note = \{note\},\\
      url = \{url\},\\
    \}
  }

  \item Para livretos sem o campo author: 

  \formato{\citetext{bib:booklet1}}

  \formato{
    @book\{bib:booklet1,\\
      editor = \{editor booklet\},\\
      type = \{type\},\\
      title = \{title booklet\},\\
      edition = \{edition\},\\
      address = \{address\},\\
      volume = \{volume\},\\
      number = \{number\},\\
      series = \{series\},\\
      month = \{month\},\\
      year = \{year\},\\
      pages = \{pages\},\\
      note = \{note\},\\
      url = \{url\},\\
    \}
  }

  \item Para livretos sem os campos author e editor: 

  \formato{\citetext{bib:booklet2}}

  \formato{
    @book\{bib:booklet2,\\
      organization = \{organization booklet\},\\
      type = \{type\},\\
      title = \{title booklet\},\\
      edition = \{edition\},\\
      address = \{address\},\\
      volume = \{volume\},\\
      number = \{number\},\\
      series = \{series\},\\
      month = \{month\},\\
      year = \{year\},\\
      pages = \{pages\},\\
      note = \{note\},\\
      url = \{url\},\\
    \}
  }

  \item Para livretos sem os campos author, editor e organization: 

  \formato{\citetext{bib:booklet3}}

  \formato{
    @book\{bib:booklet3,\\
      type = \{type\},\\
      title = \{title booklet\},\\
      edition = \{edition\},\\
      address = \{address\},\\
      volume = \{volume\},\\
      number = \{number\},\\
      series = \{series\},\\
      month = \{month\},\\
      year = \{year\},\\
      pages = \{pages\},\\
      note = \{note\},\\
      url = \{url\},\\
    \}
  }
\end{itemize}

\begin{itemize}
\item Para manual: 

\formato{\citetext{bib:manual}}

\formato{
  @manual\{bib:manual,\\
    author = \{author manual\},\\
    title = \{title manual\},\\
    edition = \{edition\},\\
    address = \{address\},\\
    month = \{month\},\\
    year = \{year\},\\
    pages = \{pages\},\\
    series = \{series\},\\
    number = \{number\},\\
    note = \{note\},\\
    url = \{url\},\\
  \}
}

\item Para manual sem campo author: 

\formato{\citetext{bib:manual1}}

\formato{
  @manual\{bib:manual1,\\
    editor = \{editor manual\},\\
    title = \{title manual\},\\
    edition = \{edition\},\\
    address = \{address\},\\
    month = \{month\},\\
    year = \{year\},\\
    pages = \{pages\},\\
    series = \{series\},\\
    number = \{number\},\\
    note = \{note\},\\
    url = \{url\},\\
  \}
}

\item Para manual sem campos author e editor: 

\formato{\citetext{bib:manual2}}

\formato{
  @manual\{bib:manual2,\\
    organization = \{organization manual\},\\
    title = \{title manual\},\\
    edition = \{edition\},\\
    address = \{address\},\\
    month = \{month\},\\
    year = \{year\},\\
    pages = \{pages\},\\
    series = \{series\},\\
    number = \{number\},\\
    note = \{note\},\\
    url = \{url\},\\
  \}
}

\item Para manual sem campos author, editor e organization: 

\formato{\citetext{bib:manual3}}

\formato{
  @manual\{bib:manual3,\\
    title = \{title manual\},\\
    edition = \{edition\},\\
    address = \{address\},\\
    month = \{month\},\\
    year = \{year\},\\
    pages = \{pages\},\\
    series = \{series\},\\
    number = \{number\},\\
    note = \{note\},\\
    url = \{url\},\\
  \}
}
\end{itemize}

\begin{itemize}
\item Para relatórios técnicos: 

\formato{\citetext{bib:techreport}}

\formato{
  @techreport\{bib:techreport,\\
    author = \{author techreport\},\\
    title = \{title techreport\},\\
    edition = \{edition\},\\
    address = \{address\},\\
    month = \{month\},\\
    year = \{year\},\\
    pages = \{pages\},\\
    series = \{series\},\\
    number = \{number\},\\
    note = \{note\},\\
    url = \{url\},\\
  \}
}

\item Para relatórios técnicos sem o campo author: 

\formato{\citetext{bib:techreport1}}

\formato{
  @techreport\{bib:techreport1,\\
    editor = \{editor techreport\},\\
    title = \{title techreport\},\\
    edition = \{edition\},\\
    address = \{address\},\\
    month = \{month\},\\
    year = \{year\},\\
    pages = \{pages\},\\
    series = \{series\},\\
    number = \{number\},\\
    note = \{note\},\\
    url = \{url\},\\
  \}
}

\item Para relatórios técnicos sem os campos author e editor: 

\formato{\citetext{bib:techreport2}}

\formato{
  @techreport\{bib:techreport2,\\
    organization = \{organization techreport\},\\
    title = \{title techreport\},\\
    edition = \{edition\},\\
    address = \{address\},\\
    month = \{month\},\\
    year = \{year\},\\
    pages = \{pages\},\\
    series = \{series\},\\
    number = \{number\},\\
    note = \{note\},\\
    url = \{url\},\\
  \}
}

\item Para relatórios técnicos sem os campos author, editor e organization: 

\formato{\citetext{bib:techreport3}}

\formato{
  @techreport\{bib:techreport3,\\
    title = \{title techreport\},\\
    edition = \{edition\},\\
    address = \{address\},\\
    month = \{month\},\\
    year = \{year\},\\
    pages = \{pages\},\\
    series = \{series\},\\
    number = \{number\},\\
    note = \{note\},\\
    url = \{url\},\\
  \}
}
\end{itemize}

\begin{itemize}
\item Para livros: 

\formato{\citetext{bib:book}}

\formato{
  @book\{bib:book,\\
    author = \{author book\},\\
    type = \{type\},\\
    title = \{title book\},\\
    edition = \{edition\},\\
    address = \{address\},\\
    publisher = \{publisher\},\\
    year = \{year\},\\
    pages = \{pages\},\\
    howpublished = \{howpublished\},\\
    series = \{series\},\\
    number = \{number\},\\
    note = \{note\},\\
    url = \{url\},\\
  \}
}

\item Para livros sem o campo author: 

\formato{\citetext{bib:book1}}

\formato{
  @book\{bib:book1,\\
    editor = \{editor book\},\\
    type = \{type\},\\
    title = \{title book\},\\
    edition = \{edition\},\\
    address = \{address\},\\
    publisher = \{publisher\},\\
    year = \{year\},\\
    pages = \{pages\},\\
    howpublished = \{howpublished\},\\
    series = \{series\},\\
    number = \{number\},\\
    note = \{note\},\\
    url = \{url\},\\
  \}
}

\item Para livros sem os campos author e editor: 

\formato{\citetext{bib:book2}}

\formato{
  @book\{bib:book2,\\
    organization = \{organization book\},\\
    type = \{type\},\\
    title = \{title book\},\\
    edition = \{edition\},\\
    address = \{address\},\\
    publisher = \{publisher\},\\
    year = \{year\},\\
    pages = \{pages\},\\
    howpublished = \{howpublished\},\\
    series = \{series\},\\
    number = \{number\},\\
    note = \{note\},\\
    url = \{url\},\\
  \}
}

\item Para livros sem os campos author, editor e organization: 

\formato{\citetext{bib:book3}}

\formato{
  @book\{bib:book3,\\
    type = \{type\},\\
    title = \{title book\},\\
    edition = \{edition\},\\
    address = \{address\},\\
    publisher = \{publisher\},\\
    year = \{year\},\\
    pages = \{pages\},\\
    howpublished = \{howpublished\},\\
    series = \{series\},\\
    number = \{number\},\\
    note = \{note\},\\
    url = \{url\},\\
  \}
}
\end{itemize}

\begin{itemize}
\item Para parte de livro: \formato{\citetext{bib:inbook}}

\formato{
  @inbook\{bib:inbook,\\
    author = \{author inbook\},\\
    title = \{title inbook\},\\
    booktitle = \{booktitle inbook\},\\
    edition = \{edition\},\\
    address = \{address\},\\
    publisher = \{publisher\},\\
    year = \{year\},\\
    series = \{series\},\\
    number = \{number\},\\
    type = \{type\},\\
    section = \{section\},\\
    pages = \{pages\},\\
    note = \{note\},\\
    url = \{url\},\\
  \}
}

\item Para parte de livro sem o campo author: \formato{\citetext{bib:inbook1}}

\formato{
  @inbook\{bib:inbook1,\\
    organization = \{organization inbook\},\\
    title = \{title inbook\},\\
    editor = \{editor inbook\},\\
    booktitle = \{booktitle inbook\},\\
    edition = \{edition\},\\
    address = \{address\},\\
    publisher = \{publisher\},\\
    year = \{year\},\\
    series = \{series\},\\
    number = \{number\},\\
    type = \{type\},\\
    section = \{section\},\\
    pages = \{pages\},\\
    note = \{note\},\\
    url = \{url\},\\
  \}
}

\item Para parte de livro sem os campos author e organization: \formato{\citetext{bib:inbook2}}

\formato{
  @inbook\{bib:inbook2,\\
    title = \{title inbook\},\\
    editor = \{editor inbook\},\\
    booktitle = \{booktitle inbook\},\\
    edition = \{edition\},\\
    address = \{address\},\\
    publisher = \{publisher\},\\
    year = \{year\},\\
    series = \{series\},\\
    number = \{number\},\\
    type = \{type\},\\
    section = \{section\},\\
    pages = \{pages\},\\
    note = \{note\},\\
    url = \{url\},\\
  \}
}

\item Para parte de livro sem os campos author, organization e editor: \formato{\citetext{bib:inbook3}}

\formato{
  @inbook\{bib:inbook3,\\
    title = \{title inbook\},\\
    booktitle = \{booktitle inbook\},\\
    edition = \{edition\},\\
    address = \{address\},\\
    publisher = \{publisher\},\\
    year = \{year\},\\
    series = \{series\},\\
    number = \{number\},\\
    type = \{type\},\\
    section = \{section\},\\
    pages = \{pages\},\\
    note = \{note\},\\
    url = \{url\},\\
  \}
}
\end{itemize}

\begin{itemize}
\item Para parte de livro com título: \formato{\citetext{bib:incollection}}

\formato{
  @incollection\{bib:incollection,\\
    author = \{author incollection\},\\
    title = \{title incollection\},\\
    editor = \{editor incollection\},\\
    booktitle = \{booktitle incollection\},\\
    edition = \{edition\},\\
    address = \{address\},\\
    publisher = \{publisher\},\\
    year = \{year\},\\
    series = \{series\},\\
    number = \{number\},\\
    type = \{type\},\\
    section = \{section\},\\
    pages = \{pages\},\\
    note = \{note\},\\
    url = \{url\},\\
  \}
}

\item Para parte de livro com título sem campo author: \formato{\citetext{bib:incollection1}}

\formato{
  @incollection\{bib:incollection1,\\
    organization = \{organization incollection\},\\
    title = \{title incollection\},\\
    editor = \{editor incollection\},\\
    booktitle = \{booktitle incollection\},\\
    edition = \{edition\},\\
    address = \{address\},\\
    publisher = \{publisher\},\\
    year = \{year\},\\
    series = \{series\},\\
    number = \{number\},\\
    type = \{type\},\\
    section = \{section\},\\
    pages = \{pages\},\\
    note = \{note\},\\
    url = \{url\},\\
  \}
}

\item Para parte de livro com título sem campos author e organization: \formato{\citetext{bib:incollection2}}

\formato{
  @incollection\{bib:incollection2,\\
    title = \{title incollection\},\\
    editor = \{editor incollection\},\\
    booktitle = \{booktitle incollection\},\\
    edition = \{edition\},\\
    address = \{address\},\\
    publisher = \{publisher\},\\
    year = \{year\},\\
    series = \{series\},\\
    number = \{number\},\\
    type = \{type\},\\
    section = \{section\},\\
    pages = \{pages\},\\
    note = \{note\},\\
    url = \{url\},\\
  \}
}

\item Para parte de livro com título sem campos author, organization e editor: \formato{\citetext{bib:incollection3}}

\formato{
  @incollection\{bib:incollection3,\\
    title = \{title incollection\},\\
    booktitle = \{booktitle incollection\},\\
    edition = \{edition\},\\
    address = \{address\},\\
    publisher = \{publisher\},\\
    year = \{year\},\\
    series = \{series\},\\
    number = \{number\},\\
    type = \{type\},\\
    section = \{section\},\\
    pages = \{pages\},\\
    note = \{note\},\\
    url = \{url\},\\
  \}
}
\end{itemize}

\begin{itemize}
\item Para artigo em proceedings: \formato{\citetext{bib:inproceedings}}

\formato{
  @inproceedings\{bib:inproceedings,\\
    author = \{author inproceedings\},\\
    title = \{title inproceedings\},\\
    editor = \{editor inproceedings\},\\
    booktitle = \{booktitle inproceedings\},\\
    address = \{address\},\\
    publisher = \{publisher\},\\
    year = \{year\},\\
    series = \{series\},\\
    number = \{number\},\\
    pages = \{pages\},\\
    note = \{note\},\\
    url = \{url\},\\
  \}
}

\item Para artigo em proceedings sem campo author: \formato{\citetext{bib:inproceedings1}}

\formato{
  @inproceedings\{bib:inproceedings1,\\
    organization = \{organization inproceedings\},\\
    title = \{title inproceedings\},\\
    editor = \{editor inproceedings\},\\
    booktitle = \{booktitle inproceedings\},\\
    address = \{address\},\\
    publisher = \{publisher\},\\
    year = \{year\},\\
    series = \{series\},\\
    number = \{number\},\\
    pages = \{pages\},\\
    note = \{note\},\\
    url = \{url\},\\
  \}
}

\item Para artigo em proceedings sem campos author e organization: \formato{\citetext{bib:inproceedings2}}

\formato{
  @inproceedings\{bib:inproceedings2,\\
    title = \{title inproceedings\},\\
    editor = \{editor inproceedings\},\\
    booktitle = \{booktitle inproceedings\},\\
    address = \{address\},\\
    publisher = \{publisher\},\\
    year = \{year\},\\
    series = \{series\},\\
    number = \{number\},\\
    pages = \{pages\},\\
    note = \{note\},\\
    url = \{url\},\\
  \}
}

\item Para artigo em proceedings sem campos author, organization e editor: \formato{\citetext{bib:inproceedings3}}

\formato{
  @inproceedings\{bib:inproceedings3,\\
    title = \{title inproceedings\},\\
    booktitle = \{booktitle inproceedings\},\\
    address = \{address\},\\
    publisher = \{publisher\},\\
    year = \{year\},\\
    series = \{series\},\\
    number = \{number\},\\
    pages = \{pages\},\\
    note = \{note\},\\
    url = \{url\},\\
  \}
}
\end{itemize}

\begin{itemize}
\item Para proceedings: \formato{\citetext{bib:proceedings}}

\formato{
  @proceedings\{bib:proceedings,\\
    editor = \{editor proceedings\},\\
    title = \{title proceedings\},\\
    volume = \{volume\},\\
    number = \{number\},\\
    series = \{series\},\\
    organization = \{organization proceedings\},\\
    address = \{address\},\\
    publisher = \{publisher\},\\
    year = \{year\},\\
    pages = \{pages\},\\
    note = \{note\},\\
    url = \{url\},\\
  \}
}

\item Para proceedings sem o campo editor: \formato{\citetext{bib:proceedings1}}

\formato{
  @proceedings\{bib:proceedings1,\\
    organization = \{organization proceedings\},\\
    title = \{title proceedings\},\\
    volume = \{volume\},\\
    number = \{number\},\\
    series = \{series\},\\
    address = \{address\},\\
    publisher = \{publisher\},\\
    year = \{year\},\\
    pages = \{pages\},\\
    note = \{note\},\\
    url = \{url\},\\
  \}
}

\item Para proceedings sem o campo editor e organization: \formato{\citetext{bib:proceedings2}}

\formato{
  @proceedings\{bib:proceedings2,\\
    title = \{title proceedings\},\\
    volume = \{volume\},\\
    number = \{number\},\\
    series = \{series\},\\
    address = \{address\},\\
    publisher = \{publisher\},\\
    year = \{year\},\\
    pages = \{pages\},\\
    note = \{note\},\\
    url = \{url\},\\
  \}
}
\end{itemize}

\begin{itemize}
\item Para misc: \formato{\citetext{bib:misc}}

\formato{
  @misc\{bib:misc,\\
    author = \{author\},\\
    type = \{type\},\\
    title = \{title\},\\
    address = \{address\},\\
    publisher = \{publisher\},\\
    month = \{month\},\\
    year = \{year\},\\
    pages = \{pages\},\\
    howpublished = \{howpublished\},\\
    series = \{series\},\\
    number = \{number\},\\
    note = \{note\},\\
    url = \{url\},\\
  \}
}

\item Para misc sem campo author: \formato{\citetext{bib:misc1}}

\formato{
  @misc\{bib:misc1,\\
    editor = \{editor\},\\
    type = \{type\},\\
    title = \{title\},\\
    address = \{address\},\\
    publisher = \{publisher\},\\
    month = \{month\},\\
    year = \{year\},\\
    pages = \{pages\},\\
    howpublished = \{howpublished\},\\
    series = \{series\},\\
    number = \{number\},\\
    note = \{note\},\\
    url = \{url\},\\
  \}
}

\item Para misc sem campos author e editor: \formato{\citetext{bib:misc2}}

\formato{
  @misc\{bib:misc2,\\
    organization = \{organization\},\\
    type = \{type\},\\
    title = \{title\},\\
    address = \{address\},\\
    publisher = \{publisher\},\\
    month = \{month\},\\
    year = \{year\},\\
    pages = \{pages\},\\
    howpublished = \{howpublished\},\\
    series = \{series\},\\
    number = \{number\},\\
    note = \{note\},\\
    url = \{url\},\\
  \}
}

\item Para misc sem campos author, editor e organization: \formato{\citetext{bib:misc3}}

\formato{
  @misc\{bib:misc3,\\
    type = \{type\},\\
    title = \{title\},\\
    address = \{address\},\\
    publisher = \{publisher\},\\
    month = \{month\},\\
    year = \{year\},\\
    pages = \{pages\},\\
    howpublished = \{howpublished\},\\
    series = \{series\},\\
    number = \{number\},\\
    note = \{note\},\\
    url = \{url\},\\
  \}
}
\end{itemize}

\begin{itemize}
\item Para unpublished: \formato{\citetext{bib:unpublished}}

\formato{
  @unpublished\{bib:unpublished,\\
    author = \{author unpublished\},\\
    title = \{title unpublished\},\\
    note = \{note\},\\
    month = \{month\},\\
    year = \{year\},\\
    url = \{url\},\\
  \}
}

\item Para unpublished: \formato{\citetext{bib:unpublished1}}

\formato{
  @unpublished\{bib:unpublished1,\\
    editor = \{editor unpublished\},\\
    title = \{title unpublished\},\\
    note = \{note\},\\
    month = \{month\},\\
    year = \{year\},\\
    url = \{url\},\\
  \}
}

\item Para unpublished: \formato{\citetext{bib:unpublished2}}

\formato{
  @unpublished\{bib:unpublished2,\\
    organization = \{organization unpublished\},\\
    title = \{title unpublished\},\\
    note = \{note\},\\
    month = \{month\},\\
    year = \{year\},\\
'    url = \{url\},\\
  \}
}

\item Para unpublished: \formato{\citetext{bib:unpublished3}}

\formato{
  @unpublished\{bib:unpublished3,\\
    title = \{title unpublished\},\\
    note = \{note\},\\
    month = \{month\},\\
    year = \{year\},\\
    url = \{url\},\\
  \}
}
\end{itemize}

%@mastersthesis{bib:mastersthesis,
%  address = {address},
%  annote = {annote},
%  author = {author},
%  booktitle = {booktitle},
%  section = {section},
%  edition = {edition},
%  editor = {editor},
%  eprint = {eprint},
%  howpublished = {howpublished},
%  institution = {institution},
%  journal = {journal},
%  key = {key},
%  month = {month},
%  note = {note},
%  number = {number},
%  organization = {organization},
%  pages = {pages},
%  publisher = {publisher},
%  school = {school},
%  series = {series},
%  title = {title},
%  type = {type},
%  url = {url},
%  volume = {volume},
%  year = {year},
%}

%@phdthesis{bib:phdthesis,
%  address = {address},
%  annote = {annote},
%  author = {author},
%  booktitle = {booktitle},
%  section = {section},
%  edition = {edition},
%  editor = {editor},
%  eprint = {eprint},
%  howpublished = {howpublished},
%  institution = {institution},
%  journal = {journal},
%  key = {key},
%  month = {month},
%  note = {note},
%  number = {number},
%  organization = {organization},
%  pages = {pages},
%  publisher = {publisher},
%  school = {school},
%  series = {series},
%  title = {title},
%  type = {type},
%  url = {url},
%  volume = {volume},
%  year = {year},
%}

%::::::::::::::::::::::::::::::::::::::::::::::::::::::::::::::::::::::::::::
\subsection{Documentos impressos no todo}
%::::::::::::::::::::::::::::::::::::::::::::::::::::::::::::::::::::::::::::

As normas da UERJ padronizam, segundo as recomendações da ABNT, o estilo de referenciamento de diferentes documentos. A tabela \ref{tab:tabela1} apresenta a relação de tipos de documentos.

\begin{table}[!ht]
  \caption{Tipos de documentos.}
  \center
  \begin{tabular}{l}
    \hline
    Tipo \\
    \hline
    Livros \\
    Bíblia \\
    Teses e dissertações \\
    Eventos \\
    Relatórios técnicos \\
    Normas técnicas \\
    Patentes \\
    Resenhas ou recensões \\
    Publicações periódicas \\
    Documentos jurídicos \\
    Documentos cartográficos \\
    Imagens em movimento \\
    Documentos iconográficos \\
    Documentos sonoros e musicais \\
    Documentos tridimensionais \\
    Entrevistas \\
    \hline
  \end{tabular}
  \label{tab:tabela1}
\end{table}

%~~~~~~~~~~~~~~~~~~~~~~~~~~~~~~~~~~~~~~~~~~~~~~~~~~~~~~
\subsubsection{Livros, folhetos, manuais, guias, catálogos, enciclopédias, dicionários}
%~~~~~~~~~~~~~~~~~~~~~~~~~~~~~~~~~~~~~~~~~~~~~~~~~~~~~~


\begin{itemize}[leftmargin=\parindent,parsep=0pt,itemsep=0pt]
  \item Formatação da referência:

  \formato{AUTOR(ES). \textsl{Título}: subtítulo. Edição. Local de publicação: 
  Editora, data de publicação. Número de páginas ou volumes.}

  \item Modelo na base de dados:

  \formato{
    @book\{rótulo,\\
    author=\{autor(es)\},\\
    title=\{título\},\\
    subtitle=\{subtítulo\},\\
    edition=\{edição\},\\
    address=\{local de publicação\},\\
    publisher=\{editora\},\\
    year=\{data de publicação\}\}\\
    pages=\{número de paginas ou volumes\},\\
  }
\end{itemize}

\begin{itemize}[label={Ex.:},leftmargin=\parindent,parsep=0pt,itemsep=0pt]
  \item \formato{\citetext{bib:Borheim1976}}

  \begin{itemize}[leftmargin=*,parsep=0pt,itemsep=0pt]
    \item Entrada na base de dados:

    \formato{
      @book\{bib:Borheim1976,\\
      author=\{Borheim, Gerd\},\\
      title=\{Introdu{\c c}{\~a}o ao filosofar\},\\
      subtitle=\{o pensamento filos{\'o}fico em bases existenciais\},\\
      edition=\{3\},\\
      address=\{Porto Alegre\},\\
      publisher=\{Ed. Globo\},\\
      year=\{1976\},
      pages=\{117\}\}\\
    }

    \item Citação direta: \citeonline{bib:Borheim1976}
    \item Citação indireta: \cite{bib:Borheim1976}
  \end{itemize}
\end{itemize}

%~~~~~~~~~~~~~~~~~~~~~~~~~~~~~~~~~~~~~~~~~~~~~~~~~~~~~~
\subsubsection{Bíblia}
%~~~~~~~~~~~~~~~~~~~~~~~~~~~~~~~~~~~~~~~~~~~~~~~~~~~~~~

\begin{itemize}[leftmargin=\parindent,parsep=0pt,itemsep=0pt]
  \item Formatação da referência:

  \formato{BÍBLIA. Língua da publicação. \textsl{Título}: subtítulo. 
  Local: Editora, data de publicação. Número de páginas ou volumes.}

  \item Entrada na base de dados:

  \formato{
    @book\{rótulo,\\
      organization=\{B{\'\i}blia\},\\
      type=\{língua da publicação\},\\
      title=\{título\},\\
      subtitle=\{subtítulo\},\\
      address=\{local\},\\
      publisher=\{editora\},\\
      year=\{data da publicação\},\\
      pages=\{número de páginas ou volumes\}\}\\
  }
\end{itemize}

\begin{itemize}[label={Ex.:},leftmargin=\parindent,parsep=0pt,itemsep=0pt]
  \item \formato{\citetext{bib:Biblia1987}}

  \begin{itemize}[leftmargin=*,parsep=0pt,itemsep=0pt]
    \item Entrada na base de dados:

    \formato{
      @book\{bib:Biblia1987,\\
      organization=\{B{\'\i}blia\},\\
      type=\{Italiano\},\\
      title=\{La Biblia\},\\
      subtitle=\{novissima versione dai testi originali\},\\
      address=\{Milano\},\\
      publisher=\{Paoline\},\\
      year=\{1987\},\\
      pages=\{320\}\}\\
    }

    \item Citação direta: \citeonline{bib:Biblia1987}
    \item Citação indireta: \cite{bib:Biblia1987}
  \end{itemize}
\end{itemize}

%~~~~~~~~~~~~~~~~~~~~~~~~~~~~~~~~~~~~~~~~~~~~~~~~~~~~~~
\subsubsection{Teses e dissertações}
%~~~~~~~~~~~~~~~~~~~~~~~~~~~~~~~~~~~~~~~~~~~~~~~~~~~~~~

\begin{itemize}[leftmargin=\parindent,parsep=0pt,itemsep=0pt]
  \item Formatação da referência:

    \formato{AUTOR. \textsl{Título}: subtítulo. Data (ano) da conclusão da 
    tese/dissertação. Número de folhas. Tipo de documento (grau e área de 
    concentração) -- Instituição, local, data da defesa mencionada na folha 
    de aprovação (se houver).}

  \item Entrada na base de dados:

  \formato{
    @phdthesis\{bib:Correa1997,\\
      author=\{autor\},\\
      title=\{título\},\\
      subtitle=\{subtítulo\},\\
      year-presented=\{ano da conclusão\}\\
      pages=\{número de folhas\},\\
      pagename=\{f.\},\\
      type=\{grau e área de concentração\},\\
      school=\{instituição\},\\
      address=\{local\},\\
      year=\{ano na folha de aprovação\}\}\\
  }
\end{itemize}

\begin{itemize}[label={Ex.:},leftmargin=\parindent,parsep=0pt,itemsep=0pt]
  \item \formato{\citetext{bib:Correa1997}}

  \begin{itemize}[leftmargin=*,parsep=0pt,itemsep=0pt]
    \item Entrada na base de dados:

    \formato{
      @phdthesis\{bib:Correa1997,\\
      author=\{Corr{\^e}a, Marilena Cordeiro Dias Villela\},\\
      title=\{A tecnologia a servi{\c c}o de um sonho\},\\
      subtitle=\{um estudo sobre a reprodu{\c c}{\~a}o assistida\},\\
      year-presented=\{1997\}\\
      pages=\{290\},\\
      pagename=\{f.\},\\
      type=\{Doutorado em Saúde Coletiva\},\\
      school=\{Instituto de Medicina Social, 
               Universidade do Estado do Rio de Janeiro\},\\
      address=\{Rio de Janeiro\},\\
      year=\{1998\}\}\\
    }

    \item Citação direta: \citeonline{bib:Correa1997}
    \item Citação indireta: \cite{bib:Correa1997}
  \end{itemize}
\end{itemize}

\begin{itemize}[label={Ex.:},leftmargin=\parindent,parsep=0pt,itemsep=0pt]
  \item \formato{\citetext{bib:Rebeca1999}}

  \begin{itemize}[leftmargin=*,parsep=0pt,itemsep=0pt]
    \item Entrada na base de dados:

    \formato{
      @mastersthesis\{bib:Rebeca1999,\\
      author          = \{Rebeca, Rosilene\},\\
      title           = \{Influência do ciclo estral no comportamento rotacional 
                          em nado livre de camundongos suíços adultos\},\\
      year-presented  = \{1999\},\\
      pages           = \{79\},\\
      pagename        = \{f.\},\\
      type            = \{Mestrado em Biologia\},\\
      school          = \{Instituto de Biologia Roberto Alcântara Gomes,
                          Universidade do Estado do Rio de Janeiro\},\\
      address         = \{Rio de Janeiro\},\\
      year            = \{1999\}\}\\
    }

    \item Citação direta: \citeonline{bib:Rebeca1999}
    \item Citação indireta: \cite{bib:Rebeca1999}
  \end{itemize}
\end{itemize}

\begin{itemize}[label={Ex.:},leftmargin=\parindent,parsep=0pt,itemsep=0pt]
  \item \formato{\citetext{bib:Morgado1990}}

  \begin{itemize}[leftmargin=*,parsep=0pt,itemsep=0pt]
    \item Entrada na base de dados:

    \formato{
      @monography\{bib:Morgado1990,\\
      author=\{M. L. C. Morgado\},\\
      title=\{Reimplante dent\'ario\},\\
      year-presented=\{1990\},\\
      pages=\{51\},\\
      pagename=\{f.\},\\
      type=\{Gradua{\c c}\~ao em Odontologia\},\\
      school=\{Faculdade de Odontologia, Universidade Camilo Castelo Branco\},\\
      address=\{S\~ao Paulo\},\\
      year=\{1990\}\}\\
    }

    \item Citação direta: \citeonline{bib:Morgado1990}
    \item Citação indireta: \cite{bib:Morgado1990}
  \end{itemize}
\end{itemize}

%~~~~~~~~~~~~~~~~~~~~~~~~~~~~~~~~~~~~~~~~~~~~~~~~~~~~~~
\subsubsection{Eventos (congressos, conferências, seminários)}
%~~~~~~~~~~~~~~~~~~~~~~~~~~~~~~~~~~~~~~~~~~~~~~~~~~~~~~

\begin{itemize}[leftmargin=\parindent,parsep=0pt,itemsep=0pt]
  \item Formatação da referência:

    \formato{NOME DO EVENTO, número (se houver), ano de realização, local 
    de realização (cidade). \textsl{Título do documento}. Local de publicação: 
    Editora, data de publicação. Número de páginas ou volumes.}

  \item Entrada na base de dados:

    \formato{
      @proceedings\{rótulo,\\
      organization = \{nome do evento\},\\
      org-short = \{abreviatura do evento\},\\
      conference-number = \{número\},\\
      conference-year = \{ano da realização\},\\
      conference-location = \{local da realização\},\\
      title = \{título do documento\},\\
      address = \{local da publicação\},\\
      publisher = \{editora\},\\
      year = \{data da publicação\},\\
      note = \{número de páginas ou volumes\}\}\\
  }
\end{itemize}

\begin{itemize}[label={Ex.:},leftmargin=\parindent,parsep=0pt,itemsep=0pt]
  \item \formato{\citetext{bib:CBA1996}}

  \begin{itemize}[leftmargin=*,parsep=0pt,itemsep=0pt]
    \item Entrada na base de dados:

    \formato{
      @proceedings\{bib:CBA1996,\\
      organization = \{Congresso Brasileiro de Autom{\'a}tica\},\\
      org-short = \{CBA'96\},\\
      conference-number = \{11\},\\
      conference-year = \{1996\},\\
      conference-location = \{S{\~a}o Paulo\},\\
      title = \{Anais...\},\\
      address = \{S{\~a}o Paulo\},\\
      publisher = \{SBA\},\\
      year = \{1996\},\\
      note = \{3 v\}\}\\
    }

    \item Citação direta: \citeonline{bib:CBA1996}
    \item Citação indireta: \cite{bib:CBA1996}
  \end{itemize}
\end{itemize}

%~~~~~~~~~~~~~~~~~~~~~~~~~~~~~~~~~~~~~~~~~~~~~~~~~~~~~~
\subsubsection{Relatórios técnicos}
%~~~~~~~~~~~~~~~~~~~~~~~~~~~~~~~~~~~~~~~~~~~~~~~~~~~~~~

\begin{itemize}[leftmargin=\parindent,parsep=0pt,itemsep=0pt]
  \item Formatação da referência:

    \formato{AUTOR(ES). \textsl{Título}: subtítulo. Edição. Local de 
    publicação: Editora, data de publicação. Número de páginas ou volumes. 
    Relatório técnico.}

  \item Entrada na base de dados:

  \formato{
    @book\{rótulo,\\
    author        = \{autor(es)\},\\
    title         = \{título\},\\
    subtitle      = \{subtítulo\},\\
    edition       = \{edição\},\\
    address       = \{local de publicação\},\\
    publisher     = \{editora\},\\
    year          = \{data de publicação\},\\
    pages         = \{número de páginas ou volumes\},\\
    howpublished  = \{Relat{\'o}rio t{\'e}cnico\}\}\\
  }
\end{itemize}

\begin{itemize}[label={Ex.:},leftmargin=\parindent,parsep=0pt,itemsep=0pt]
  \item \formato{\citetext{bib:Silva1985}}

  \begin{itemize}[leftmargin=*,parsep=0pt,itemsep=0pt]
    \item Entrada na base de dados:

    \formato{
      @book\{bib:Silva1985,\\
      author = \{Silva, L. S.\},\\
      title = \{Manuten{\c c}{\~a}o de softwares\},\\
      address = \{Campinas\},\\
      publisher = \{UNICAMP-FEE-DCA\},\\
      year = \{1985\},\\
      pages = \{110\},\\
      howpublished = \{Relat{\'o}rio t{\'e}cnico\}\}\\
    }

    \item Citação direta: \citeonline{bib:Silva1985}
    \item Citação indireta: \cite{bib:Silva1985}
  \end{itemize}
\end{itemize}

%~~~~~~~~~~~~~~~~~~~~~~~~~~~~~~~~~~~~~~~~~~~~~~~~~~~~~~
\subsubsection{Normas técnicas}
%~~~~~~~~~~~~~~~~~~~~~~~~~~~~~~~~~~~~~~~~~~~~~~~~~~~~~~

\begin{itemize}[leftmargin=\parindent,parsep=0pt,itemsep=0pt]
  \item Formatação da referência:

    \formato{ENTIDADE RESPONSÁVEL. \textsl{Título da norma}: subtítulo. 
    Local de publicação, data de publicação. Número de páginas.}

  \item Entrada na base de dados:

  \formato{
    @manual\{rótulo,\\
    organization = \{entidade responsável\},\\
    org-short = \{abreviatura da entidade\},\\
    title = \{título da norma\},\\
    subtitle = \{subtítulo\},\\
    address = \{local de publicação\},\\
    year = \{data de publicação\},\\
    pages = \{número de páginas\}\}\\
  }
\end{itemize}

\begin{itemize}[label={Ex.:},leftmargin=\parindent,parsep=0pt,itemsep=0pt]
  \item \formato{\citetext{bib:NBR6023}}

  \begin{itemize}[leftmargin=*,parsep=0pt,itemsep=0pt]
    \item Entrada na base de dados:

    \formato{
      @manual\{bib:nbr6023,\\
      organization = \{associa{\c c}\~ao brasileira de normas t\'ecnicas\},\\
      org-short = \{ABNT\},\\
      title = \{\{NBR\} 6023\},\\
      subtitle = \{informa{\c c}\~ao e documenta{\c c}\~ao: referências: 
                   elabora{\c c}\~ao\},\\
      address = \{rio de janeiro\},\\
      year = \{2002\},\\
      pages = \{24\}\}\\
    }

    \item Citação direta: \citeonline{bib:NBR6023}
    \item Citação indireta: \cite{bib:NBR6023}
  \end{itemize}
\end{itemize}

%~~~~~~~~~~~~~~~~~~~~~~~~~~~~~~~~~~~~~~~~~~~~~~~~~~~~~~
\subsubsection{Patentes}
%~~~~~~~~~~~~~~~~~~~~~~~~~~~~~~~~~~~~~~~~~~~~~~~~~~~~~~

\begin{itemize}[leftmargin=\parindent,parsep=0pt,itemsep=0pt]
  \item Formatação da referência:

    \formato{ENTIDADE RESPONSÁVEL (se houver). AUTOR(ES) na ordem direta 
    de seus nomes separados por ponto e vírgula. \textsl{Título}. Número 
    da patente, data do depósito, data da concessão.}

  \item Entrada na base de dados:

    \formato{
    @patent\{rótulo,\\
      organization = \{entidade responsável\},\\
      author = \{autor(es)\},\\
      title = \{título\},\\
      number = \{número da patente\},\\
      howpublished = \{data de depósito, data da concessão\}\}\\
    }
\end{itemize}

\begin{itemize}[label={Ex.:},leftmargin=\parindent,parsep=0pt,itemsep=0pt]
  \item \formato{\citetext{bib:Nabisco1984}}

  \begin{itemize}[leftmargin=*,parsep=0pt,itemsep=0pt]
    \item Entrada na base de dados:

    \formato{
      @patent\{bib:Nabisco1984,\\
      organization = \{Nabisco Brands, Inc\},\\
      author = \{P. O. Horwart; P. M. Irbe\},\\
      title = \{Process for preparing fructuose from starch\},\\
      number = \{US n. 4.458.017\},\\
      howpublished = \{30 jun. 1982, 3 jul. 1984\}\}\\
    }

    \item Citação direta: \citeonline{bib:Nabisco1984}
    \item Citação indireta: \cite{bib:Nabisco1984}\\
  \end{itemize}

  \item \formato{\citetext{bib:Huntington1998}}

  \begin{itemize}[leftmargin=*,parsep=0pt,itemsep=0pt]
    \item Entrada na base de dados:

    \formato{
      @patent\{bib:Huntington1998,\\
      organization = \{Huntington Medical Research Institutes\},\\
      author = \{John Albert Arcadi\},\\
      title = \{Composi{\c c}{\~a}o e m{\'e}todo para tratamento de 
               c{\^a}ncer de pr{\'o}stata\},\\
      number = \{BR n. PI 9603454-8\},\\
      howpublished = \{16 ago. 1996, 12 maio 1998\}\}\\
    }

    \item Citação direta: \citeonline{bib:Huntington1998}
    \item Citação indireta: \cite{bib:Huntington1998}
  \end{itemize}
\end{itemize}

%~~~~~~~~~~~~~~~~~~~~~~~~~~~~~~~~~~~~~~~~~~~~~~~~~~~~~~
\subsubsection{Resenha ou Recensão}
%~~~~~~~~~~~~~~~~~~~~~~~~~~~~~~~~~~~~~~~~~~~~~~~~~~~~~~

\begin{itemize}[leftmargin=\parindent,parsep=0pt,itemsep=0pt]
  \item Formatação da referência:

    \formato{AUTOR(ES) da publicação resenhada. Título da publicação resenhada. 
    Edição. Local de  publicação: Editora, data. Número de páginas. Resenha de: 
    AUTOR da resenha. Título da resenha e demais dados da publicação que trouxe 
    a resenha.}

  \item Entrada na base de dados:

    \formato{
      @book\{rótulo,\\
      author = \{autor(es)\},\\
      title = \{título da publicação resenhada\},\\
      edition = \{edição\},\\
      address = \{local de publicação\},\\
      publisher = \{editora\},\\
      year = \{data\},\\
      pages = \{número de páginas\},\\
      note = \{Resenha de: \textbackslash citetext\{referência da autoria da resenha\}\}\\
    }
\end{itemize}

\begin{itemize}[label={Ex.:},leftmargin=\parindent,parsep=0pt,itemsep=0pt]
  \item \formato{\citetext{bib:Veloso1998}}

%        \formato{\citetext{bib:Neumane1998}}

  \begin{itemize}[leftmargin=*,parsep=0pt,itemsep=0pt]
    \item Entrada na base de dados:

    \formato{
      @book\{bib:Veloso1998,\\
      author = \{Veloso, Caetano\},\\
      title = \{Verdade tropical\},\\
      address = \{S{\~a}o Paulo\},\\
      publisher = \{Cia das Letras\},\\
      year = \{1998\},\\
      pages = \{524\},\\
      note = \{Resenha de: \textbackslash citetext\{bib:Neumane1998\}\}\\
    }

    \formato{
      @article\{bib:Neumane1998,\\
      title = \{Caetano\},\\
      subtitle = \{lendo nas entrelinhas\},\\
      author = \{Neumane, Jos{\'e}\},\\
      journal = \{Livro Aberto\},\\
      number = \{10\},\\
      volume = \{2\},\\
      address = \{S{\~a}o Paulo\},\\
      month = \{nov.\},\\
      year = \{1998\},\\
      note = \{\{p\}.~15--16\}\}\\
    }

    \item Citação direta: \citeonline{bib:Veloso1998}
    \item Citação indireta: \cite{bib:Veloso1998}
  \end{itemize}
\end{itemize}

%~~~~~~~~~~~~~~~~~~~~~~~~~~~~~~~~~~~~~~~~~~~~~~~~~~~~~~
\subsubsection{Publicações periódicas (revistas, boletins, anuários, etc.)}
%~~~~~~~~~~~~~~~~~~~~~~~~~~~~~~~~~~~~~~~~~~~~~~~~~~~~~~

\begin{itemize}[leftmargin=\parindent,parsep=0pt,itemsep=0pt]
  \item Formatação da referência:

    \formato{TÍTULO DO PERIÓDICO. Local de publicação: Editora, data (ano) 
    do primeiro volume seguido de hífen e, se a publicação cessou, data (ano) 
    do último volume. Periodicidade.}

  \item Entrada na base de dados:

    \formato{
      @journalpart\{rótulo,\\
      title = \{título do periódico\},\\
      address = \{local da publicação\},\\
      publisher = \{editora\},\\
      year = \{data do primeiro volume-data do último volume\},\\
      note = \{periodicidade\}\}\\
    }
\end{itemize}

\begin{itemize}[label={Ex.:},leftmargin=\parindent,parsep=0pt,itemsep=0pt]
  \item \formato{\citetext{bib:AGEV1968}}

  \begin{itemize}[leftmargin=*,parsep=0pt,itemsep=0pt]
    \item Entrada na base de dados:

    \formato{
      @journalpart\{bib:AGEV1968,\\
      title = \{Anu{\'a}rio Internacional\},\\
      address = \{S{\~a}o Paulo\},\\
      publisher = \{AGEV\},\\
      year = \{1997-1978\}\}\\
    }

    \item Citação direta: \citeonline{bib:AGEV1968}
    \item Citação indireta: \cite{bib:AGEV1968}\\
  \end{itemize}

  \item \formato{\citetext{bib:Bibliophilos1968}}

  \begin{itemize}[leftmargin=*,parsep=0pt,itemsep=0pt]
    \item Entrada na base de dados:

    \formato{
      @journalpart\{bib:Bibliophilos1968,\\
      title = \{Boletim da Sociedade de Bibliophilos Barbosa Machado\},\\
      address = \{Lisboa\},\\
      publisher = \{Impr. Libano da Silva\},\\
      year = \{1910-\{ \}\{ \}\{ \}\{ \}\},\\
      note = \{Irregular\}\}\\
    }

    \item Citação direta: \citeonline{bib:Bibliophilos1968}
    \item Citação indireta: \cite{bib:Bibliophilos1968}\\
  \end{itemize}

  \item \formato{\citetext{bib:RevistaRio1985}}

  \begin{itemize}[leftmargin=*,parsep=0pt,itemsep=0pt]
    \item Entrada na base de dados:

    \formato{
      @journalpart\{bib:RevistaRio1985,\\
      title = \{Revista Rio de Janeiro\},\\
      address = \{Niter{\'o}i\},\\
      publisher = \{EDUFF\},\\
      year = \{1985-\{ \}\{ \}\{ \}\{ \}\},\\
      note = \{Quadrimestral\}\}\\
    }

    \item Citação direta: \citeonline{bib:RevistaRio1985}
    \item Citação indireta: \cite{bib:RevistaRio1985}
  \end{itemize}
\end{itemize}

%~~~~~~~~~~~~~~~~~~~~~~~~~~~~~~~~~~~~~~~~~~~~~~~~~~~~~~
\subsubsection{Documento jurídico}
%~~~~~~~~~~~~~~~~~~~~~~~~~~~~~~~~~~~~~~~~~~~~~~~~~~~~~~

Falta construir o texto...

%~~~~~~~~~~~~~~~~~~~~~~~~~~~~~~~~~~~~~~~~~~~~~~~~~~~~~~
\subsubsection{Documento cartográfico (mapa, atlas, globo, fotografia aérea etc.)}
%~~~~~~~~~~~~~~~~~~~~~~~~~~~~~~~~~~~~~~~~~~~~~~~~~~~~~~

\begin{itemize}[leftmargin=\parindent,parsep=0pt,itemsep=0pt]
  \item Formatação da referência:

    \formato{AUTOR (se houver). \textsl{Título}: subtítulo (se houver). 
    Local de publicação: Editora, data de publicação. Designação específica. 
    Escala.}

  \item Entrada na base de dados:

    \formato{
      @book\{rótulo,\\
      author = \{autor(es)\},\\
      title = \{título\},\\
      subtitle = \{subtítulo\},\\
      address = \{local de publicação\},\\
      publisher = \{editora\},\\
      year = \{data de publicação\},\\
      note = \{designação específica. escala\}\}\\
    }

    \formato{
      @manual\{rótulo,\\
      organization = \{entidade\},\\
      org-short = \{abreviatura da entidade\},\\
      title = \{título\},\\
      subtitle = \{subtítulo\},\\
      address = \{local de publicação\},\\
      year = \{data de publicação\},\\
      note = \{designação específica. escala\}\}\\
    }
\end{itemize}

\begin{itemize}[label={Ex.:},leftmargin=\parindent,parsep=0pt,itemsep=0pt]
  \item \formato{\citetext{bib:SMMSRJ1977}}

  \begin{itemize}[leftmargin=*,parsep=0pt,itemsep=0pt]
    \item Entrada na base de dados:

    \formato{
      @manual\{bib:SMMSRJ1977,\\
      organization = \{Rio De Janeiro (RJ). \{Secretaria Municipal de 
                       Meio Ambiente\}\},\\
      org-short = \{Secretaria Municipal de Meio Ambiente\},\\
      title = \{Mapa da cobertura vegetal e uso das terras\},\\
      address = \{Rio de Janeiro\},\\
      year = \{1977\},\\
      note = \{1 mapa, color. Escala 1:75.000\}\}\\
    }

    \item Citação direta: \citeonline{bib:SMMSRJ1977}
    \item Citação indireta: \cite{bib:SMMSRJ1977}\\
  \end{itemize}

  \item \formato{\citetext{bib:IBGE1959}}

  \begin{itemize}[leftmargin=*,parsep=0pt,itemsep=0pt]
    \item Entrada na base de dados:

    \formato{
      @manual\{bib:IBGE1959,\\
      organization = \{Instituto Brasileiro de Geografia e Estat{\'\i}stica\},\\
      org-short = \{IBGE\},\\
      title = \{Atlas do Brasil\},\\
      subtitle = \{geral e regional\},\\
      address = \{Rio de Janeiro\},\\
      year = \{1959\},\\
      pages = \{705\}\}\\
    }

    \item Citação direta: \citeonline{bib:IBGE1959}
    \item Citação indireta: \cite{bib:IBGE1959}\\
  \end{itemize}

  \item \formato{\citetext{bib:Brueckmann1900}}

  \begin{itemize}[leftmargin=*,parsep=0pt,itemsep=0pt]
    \item Entrada na base de dados:

    \formato{
      @book\{bib:Brueckmann1900,\\
      author = \{Brueckmann, Gustav\},\\
      title = \{Globo\},\\
      address = \{Chicago\},\\
      publisher = \{Repogle Globes\},\\
      year = \{\{[\}19\{-\}\{-\}\{]\}\},\\
      note = \{1 globo, color. Escala: 1:41.849\}\}\\
    }

    \item Citação direta: \citeonline{bib:Brueckmann1900}
    \item Citação indireta: \cite{bib:Brueckmann1900}
  \end{itemize}
\end{itemize}


%~~~~~~~~~~~~~~~~~~~~~~~~~~~~~~~~~~~~~~~~~~~~~~~~~~~~~~
\subsubsection{Imagem em movimento (filmes, fitas de vídeo, DVD etc.)}
%~~~~~~~~~~~~~~~~~~~~~~~~~~~~~~~~~~~~~~~~~~~~~~~~~~~~~~

\begin{itemize}[leftmargin=\parindent,parsep=0pt,itemsep=0pt]
  \item Formatação da referência:

    \formato{TÍTULO: subtítulo (se houver). Créditos (diretor, produtor, 
    coordenador etc.). Elenco, se relevante. Local de publicação: Produtora, 
    data. Especificação do suporte em unidades físicas, duração, sistema de 
    reprodução, indicadores de som e cor e outras informações relevantes.}

  \item Entrada na base de dados:

    \formato{
      @book\{rótulo,\\
      title = \{título\},\\
      furtherresp = \{créditos. elenco\},\\
      address = \{local\},\\
      publisher = \{produtora\},\\
      year = \{data\},\\
      howpublished = \{informações relevantes\}\}\\
    }
\end{itemize}

\begin{itemize}[label={Ex.:},leftmargin=\parindent,parsep=0pt,itemsep=0pt]
  \item     \formato{\citetext{bib:LookFilmes1994}}

  \begin{itemize}[leftmargin=*,parsep=0pt,itemsep=0pt]
    \item Entrada na base de dados:

    \formato{
      @book\{bib:LookFilmes1994,\\
      title = \{A~liberdade {\'e} azul\},\\
      furtherresp = \{Dire{\c c}{\~a}o de Krzysztof Kieslowski\},\\
      address = \{S{\~a}o Paulo\},\\
      publisher = \{Look Filmes\},\\
      year = \{1994\},\\
      howpublished = \{1 fita de vídeo (97min), VHS, son., color., legendado\}\}\\
    }

    \item Citação direta: \citeonline{bib:LookFilmes1994}
    \item Citação indireta: \cite{bib:LookFilmes1994}
  \end{itemize}
\end{itemize}

%~~~~~~~~~~~~~~~~~~~~~~~~~~~~~~~~~~~~~~~~~~~~~~~~~~~~~~
\subsubsection{Documento iconográfico (pinturas, gravuras, fotografias etc.)}
%~~~~~~~~~~~~~~~~~~~~~~~~~~~~~~~~~~~~~~~~~~~~~~~~~~~~~~

\begin{itemize}[leftmargin=\parindent,parsep=0pt,itemsep=0pt]
  \item Formatação da referência:

    \formato{AUTOR (se houver). \textsl{Título}. Data. Especificação 
    do suporte.}

  \item Entrada na base de dados:

    \formato{
      @misc\{rótulo,\\
      author = \{autor\},\\
      title=\{título\},\\
      year=\{data\},\\
      howpublished=\{especificação do suporte\}\}\\
    }
\end{itemize}

\begin{itemize}[label={Ex.:},leftmargin=\parindent,parsep=0pt,itemsep=0pt]
  \item     \formato{\citetext{bib:Cardoso1989}}

  \begin{itemize}[leftmargin=*,parsep=0pt,itemsep=0pt]
    \item Entrada na base de dados:

    \formato{
      @misc\{bib:Cardoso1989,\\
      author = \{Cardoso, Claudio\},\\
      title=\{Pedra de Itapuca\},\\
      year=\{1989\},\\
      howpublished=\{3 fotografias, color\}\}\\
    }

    \item Citação direta: \citeonline{bib:Cardoso1989}
    \item Citação indireta: \cite{bib:Cardoso1989}\\
  \end{itemize}

  \item     \formato{\citetext{bib:Vasconcelos1988}}

  \begin{itemize}[leftmargin=*,parsep=0pt,itemsep=0pt]
    \item Entrada na base de dados:

    \formato{
      @misc\{bib:Vasconcelos1988,\\
      author = \{Vasconcelos, K\},\\
      title=\{[Sem t{\'\i}tulo]\},\\
      year=\{1988\},\\
      howpublished=\{1 fotografia\}\}\\
    }

    \item Citação direta: \citeonline{bib:Vasconcelos1988}
    \item Citação indireta: \cite{bib:Vasconcelos1988}
  \end{itemize}
\end{itemize}

%~~~~~~~~~~~~~~~~~~~~~~~~~~~~~~~~~~~~~~~~~~~~~~~~~~~~~~
\subsubsection{Documento sonoro e musical (fita cassete, CD, disco etc.)}
%~~~~~~~~~~~~~~~~~~~~~~~~~~~~~~~~~~~~~~~~~~~~~~~~~~~~~~

\begin{itemize}[leftmargin=\parindent,parsep=0pt,itemsep=0pt]
  \item Formatação da referência:

    \formato{COMPOSITOR ou INTÉRPRETE. \textsl{Título}: subtítulo (quanto 
    houver). Local: Gravadora, data. Especificação do suporte em 
    características físicas e duração.}

  \item Entrada na base de dados:

    \formato{
      @misc\{rótulo,\\
      author = \{compositor ou interprete\},\\
      title = \{título\},\\
      subtitle = \{subtítulo\},\\
      address = \{local\},\\
      publisher = \{gravadora\},\\
      year = \{data\},\\
      howpublished = \{especificação do suporte (duração)\}\}\\
    }
\end{itemize}

\begin{itemize}[label={Ex.:},leftmargin=\parindent,parsep=0pt,itemsep=0pt]
  \item     \formato{\citetext{bib:Lee1983}}

  \begin{itemize}[leftmargin=*,parsep=0pt,itemsep=0pt]
    \item Entrada na base de dados:

    \formato{
      @misc\{bib:Lee1983,\\
      author = \{Lee, Rita and Carvalho, Roberto de\},\\
      title = \{Bombom\},\\
      address = \{Rio de Janeiro\},\\
      publisher = \{Som Livre\},\\
      year = \{1983\},\\
      howpublished = \{1 fita cassete (37min), 3 3/4 pps., est{\'e}reo\}\}\\
    }

    \item Citação direta: \citeonline{bib:Lee1983}
    \item Citação indireta: \cite{bib:Lee1983}\\
  \end{itemize}

  \item     \formato{\citetext{bib:Paganini1985}}

  \begin{itemize}[leftmargin=*,parsep=0pt,itemsep=0pt]
    \item Entrada na base de dados:

    \formato{
      @misc\{bib:Paganini1985,\\
      author = \{Paganini{ }Ensemble\},\\
      title = \{Smoke gets in your eyes\},\\
      address = \{T{\'o}quio\},\\
      publisher = \{Nippon Columbia\},\\
      year = \{1985\},\\
      howpublished = \{1 CD (30min)\}\}\\
    }

    \item Citação direta: \citeonline{bib:Paganini1985}
    \item Citação indireta: \cite{bib:Paganini1985}\\
  \end{itemize}

  \item     \formato{\citetext{bib:Segovia1977}}

  \begin{itemize}[leftmargin=*,parsep=0pt,itemsep=0pt]
    \item Entrada na base de dados:

    \formato{
      @misc\{bib:Segovia1977,\\
      author = \{Segovia, Andr{\'e}s\},\\
      title = \{Bach\},\\
      subtitle = \{chaconne\},\\
      address = \{Rio de Janeiro\},\\
      publisher = \{MCA Records\},\\
      year = \{1977\},\\
      howpublished = \{1 disco sonoro, 33rpm, est{\'e}reo\}\}\\
    }

    \item Citação direta: \citeonline{bib:Segovia1977}
    \item Citação indireta: \cite{bib:Segovia1977}
  \end{itemize}
\end{itemize}

%~~~~~~~~~~~~~~~~~~~~~~~~~~~~~~~~~~~~~~~~~~~~~~~~~~~~~~
\subsubsection{Documento tridimensional (esculturas, maquetes etc.)}
%~~~~~~~~~~~~~~~~~~~~~~~~~~~~~~~~~~~~~~~~~~~~~~~~~~~~~~

\begin{itemize}[leftmargin=\parindent,parsep=0pt,itemsep=0pt]
  \item Formatação da referência:

    \formato{AUTOR(ES). \textsl{Título}: subtítulo (se houver). Data. 
    Características físicas (especificação do objeto, materiais, técnicas, 
    dimensões etc.).}

  \item Entrada na base de dados:

    \formato{
      @misc\{rótulo,\\
      author = \{autor(es)\},\\
      title = \{título\},\\
      subtitle = \{subtítulo\},\\
      year = \{data\},\\
      howpublished = \{características físicas\}\}\\
    }
\end{itemize}

\begin{itemize}[label={Ex.:},leftmargin=\parindent,parsep=0pt,itemsep=0pt]
  \item     \formato{\citetext{bib:Buonarroti1504}}

  \begin{itemize}[leftmargin=*,parsep=0pt,itemsep=0pt]
    \item Entrada na base de dados:

    \formato{
      @misc\{bib:Buonarroti1504,\\
      author = \{Buonarroti, Michelangelo\},\\
      title = \{David\},\\
      year = \{1504\},\\
      howpublished = \{Escultura renascentista, em m{\'a}rmore, com o 
                    predomínio das linhas curvas, 5,17m\}\}\\
    }

    \item Citação direta: \citeonline{bib:Buonarroti1504}
    \item Citação indireta: \cite{bib:Buonarroti1504}
  \end{itemize}
\end{itemize}

%~~~~~~~~~~~~~~~~~~~~~~~~~~~~~~~~~~~~~~~~~~~~~~~~~~~~~~
\subsubsection{Entrevistas}
%~~~~~~~~~~~~~~~~~~~~~~~~~~~~~~~~~~~~~~~~~~~~~~~~~~~~~~

\begin{itemize}[leftmargin=\parindent,parsep=0pt,itemsep=0pt]
  \item Formatação da referência para entrevista não publicada:

    \formato{
      NOME DO ENTREVISTADO. Entrevista concedida a... (nome do entrevistador). 
      Local onde foi realizada, data da realização (dia, mês abreviado e ano).
      Informações adicionais (se houver)
    }

  \item Entrada na base de dados:

    \formato{
      @manual\{rótulo,\\
      author = \{nome do entrevistado\},\\
      furtherresp = \{Entrevista concedida a (nome do entrevistador)\},\\
      address = \{local onde foi realizada a entrevista\},\\
      month = \{dia e mês da entrevista\},\\
      year = \{ano da entrevista\},\\
      note = \{informações adicionais\}\}\\
    }
\end{itemize}

\begin{itemize}[label={Ex.:},leftmargin=\parindent,parsep=0pt,itemsep=0pt]
  \item     \formato{\citetext{bib:Martins1985}}

  \begin{itemize}[leftmargin=*,parsep=0pt,itemsep=0pt]
    \item Entrada na base de dados:

    \formato{
      @manual\{bib:Martins1985,\\
      author = \{Martins, M\},\\
      furtherresp = \{Entrevista concedida a Paulo Jorge Silva\},\\
      address = \{S{\~a}o Paulo\},\\
      month = \{10 jan.\},\\
      year = \{1985\}\}\\
    }

    \item Citação direta: \citeonline{bib:Martins1985}
    \item Citação indireta: \cite{bib:Martins1985}\\
  \end{itemize}

  \item     \formato{\citetext{bib:Ferreira2006}}

  \begin{itemize}[leftmargin=*,parsep=0pt,itemsep=0pt]
    \item Entrada na base de dados:

    \formato{
      @manual\{bib:Ferreira2006,\\
      author = \{Ferreira, Carlos\},\\
      furtherresp = \{Entrevista concedida a Maria Helena de Souza\},\\
      address = \{Rio de Janeiro\},\\
      month = \{23 out.\},\\
      year = \{2006\},\\
      note = \{1 cassete sonoro (20min)\}\}\\
    }

    \item Citação direta: \citeonline{bib:Ferreira2006}
    \item Citação indireta: \cite{bib:Ferreira2006}\\
  \end{itemize}
\end{itemize}

\begin{itemize}[leftmargin=\parindent,parsep=0pt,itemsep=0pt]
  \item Formatação da referência para entrevista publicada:

    \formato{
      NOME DO ENTREVISTADO. Título da entrevista. \textsl{Título da publicação}, local 
      de publicação, número do volume ou ano (se houver), número do fascículo, data da 
      realização da entrevista (mês abreviado). Página inicial e final. Nota de 
      entrevista.
    }

  \item Entrada na base de dados:

    \formato{
      @article\{rótulo,\\
      author = \{nome do entrevistado\},\\
      title = \{título da entrevista\},\\
      journal = \{título da publicação\},\\
      address = \{local da publicação\},\\
      volume = \{número do volume ou ano\},\\
      number = \{número do fascículo\},\\
      month = \{dia, mês da entrevista\},\\
      year = \{ano da entrevista\},\\
      pages = \{página inicial - página final\},\\
      note = \{nota da entrevista\}\}\\
    }
\end{itemize}

\begin{itemize}[label={Ex.:},leftmargin=\parindent,parsep=0pt,itemsep=0pt]
  \item     \formato{\citetext{bib:Fiuza1990}}

  \begin{itemize}[leftmargin=*,parsep=0pt,itemsep=0pt]
    \item Entrada na base de dados:

    \formato{
      @article\{bib:Fiuza1990,\\
      author = \{Fiuza, Ricardo\},\\
      title = \{O~ponta-de-lan{\c c}a\},\\
      journal = \{Veja\},\\
      address = \{S{\~a}o Paulo\},\\
      number = \{1124\},\\
      month = \{04 abr.\},\\
      year = \{1990\},\\
      pages = \{9-13\},\\
      note = \{Entrevista\}\}\\
    }

    \item Citação direta: \citeonline{bib:Fiuza1990}
    \item Citação indireta: \cite{bib:Fiuza1990}
  \end{itemize}
\end{itemize}

%------------------------------------------------------------------
\subsection{Partes de documentos}
%------------------------------------------------------------------

Falta construir o texto introdutório...

%~~~~~~~~~~~~~~~~~~~~~~~~~~~~~~~~~~~~~~~~~~~~~~~~~~~~~~
\subsubsection{Partes de monografias (capítulo, volume etc.) com autoria e/ou títulos próprios}
%~~~~~~~~~~~~~~~~~~~~~~~~~~~~~~~~~~~~~~~~~~~~~~~~~~~~~~

\begin{itemize}[leftmargin=\parindent,parsep=0pt,itemsep=0pt]
  \item Formatação da referência:

  \formato{
    AUTOR(ES) DA PARTE. Título da parte. In: AUTOR(ES) DA OBRA. \textsl{Título da 
    obra}. Edição. Local de publicação: Editora, data de publicação. Identificação 
    da parte referenciada (número do capítulo e/ou volume, se houver), páginas 
    inicial e final da parte referenciada.
  }

  \item Entrada na base de dados:

  \formato{
    @incollection\{rótulo,\\
      author = \{autor(es) da parte\},\\
      title = \{título da parte\},\\
      editor = \{autor(es) da obra\},\\
      editortype = \{Org.\},\\
      booktitle = \{título da obra\},\\
      address = \{local de publicação\},\\
      publisher = \{editora\},\\
      year = \{data de publicação\},\\
      volume = \{volume referenciado\},\\
      section = \{capítulo referenciado\},\\
      pages = \{página inicial--página fina da parte\}\}\\
  }
\end{itemize}

\begin{itemize}[label={Ex.:},leftmargin=\parindent,parsep=0pt,itemsep=0pt]
  \item \formato{\citetext{bib:Oliveira1986}}

  \begin{itemize}[leftmargin=*,parsep=0pt,itemsep=0pt]
    \item Entrada na base de dados:

    \formato{
      @Incollection\{bib:Oliveira1986,\\
        author = \{Oliveira, Jo{\~a}o Batista Ara{\'u}jo e\},\\
        title = \{A organização da universidade para a pesquisa\},\\
        editor = \{Schwartzman, Simon and Castro, Claudio de Moura\},\\
        editortype = \{Org.\},\\
        booktitle = \{Pesquisa universit{\'a}ria em quest{\~a}o\},\\
        address = \{S{\~a}o Paulo\},\\
        publisher = \{Icone Ed.\},\\
        year = \{1986\},\\
        section = \{3\},\\
        pages = \{53-60\}\}\\
    }

    \item Citação direta: \citeonline{bib:Oliveira1986}
    \item Citação indireta: \cite{bib:Oliveira1986}\\
  \end{itemize}

  \item \formato{\citetext{bib:Spoerri1988}}

  \begin{itemize}[leftmargin=*,parsep=0pt,itemsep=0pt]
    \item Entrada na base de dados:

    \formato{
      @Inbook\{bib:Spoerri1988,\\
        author = \{Spoerri, T. A.\},\\
        title = \{Rea{\c c}{\~o}es psicog{\^e}nicas e neuroses\},\\
        editor = \{Spoerri, T. A.\},\\
        booktitle = \{manual de psiquiatria\},\\
        booksubtitle = \{fundamentos da cl{\'\i}nica psiqui{\'a}trica\},\\
        edition = \{8\},\\
        address = \{Rio de Janeiro\},\\
        publisher = \{Atheneu\},\\
        year = \{1988\},\\
        pages = \{159-172\}\}\\
    }

    \item Citação direta: \citeonline{bib:Spoerri1988}
    \item Citação indireta: \cite{bib:Spoerri1988}
  \end{itemize}
\end{itemize}

%~~~~~~~~~~~~~~~~~~~~~~~~~~~~~~~~~~~~~~~~~~~~~~~~~~~~~~
\subsubsection{Partes de obras (volume, tomo ou parte específicos) sem autoria especial}
%~~~~~~~~~~~~~~~~~~~~~~~~~~~~~~~~~~~~~~~~~~~~~~~~~~~~~~

\begin{itemize}[leftmargin=\parindent,parsep=0pt,itemsep=0pt]
  \item Formatação da referência:

  \formato{
    AUTOR(ES) DA OBRA. \textsl{Título da obra}. Edição. Local de publicação: Editora, 
    data de publicação. Número de volumes da obra. Número do volume, tomo ou parte que 
    se quer referenciar: Título do volume, tomo ou parte que se quer referenciar.
  }

  \item Entrada na base de dados:

  \formato{
    @misc\{rótulo,\\
      author = \{autor(es) da obra\},\\
      title = \{título da obra\},\\
      edition = \{edição\},\\
      address = \{local de publicação\},\\
      publisher = \{editora\},\\
      year = \{data de publicação\},\\
      note = \{número de volumes. volume ou tomo ou parte referenciada:
               título do volume ou tomo ou parte referenciada\}\}\\
  }
\end{itemize}

\begin{itemize}[label={Ex.:},leftmargin=\parindent,parsep=0pt,itemsep=0pt]
  \item \formato{\citetext{bib:Soares1972}}

  \begin{itemize}[leftmargin=*,parsep=0pt,itemsep=0pt]
    \item Entrada na base de dados:

    \formato{
      @misc\{bib:Soares1972,\\
        author = \{Soares, Fernandes and Burlamaqui, Carlos Kopke\},\\
        title = \{Pesquisas brasileiras, 1. e 2. graus\},\\
        address = \{S{\~a}o Paulo\},\\
        publisher = \{Formar\},\\
        year = \{1972\},\\
        note = \{3 v. V. 3: Dados estat{\'\i}sticos, microrregi{\~o}es\}\}\\
    }

    \item Citação direta: \citeonline{bib:Soares1972}
    \item Citação indireta: \cite{bib:Soares1972}
  \end{itemize}
\end{itemize}

%~~~~~~~~~~~~~~~~~~~~~~~~~~~~~~~~~~~~~~~~~~~~~~~~~~~~~~
\subsubsection{Partes de Bíblia}
%~~~~~~~~~~~~~~~~~~~~~~~~~~~~~~~~~~~~~~~~~~~~~~~~~~~~~~

\begin{itemize}[leftmargin=\parindent,parsep=0pt,itemsep=0pt]
  \item Formatação da referência:

  \formato{
    BÍBLIA. A.T. (ou N.T.). Título da parte. Idioma. \textsl{Título da obra}. 
    Edição. Local de publicação: Editora, data de publicação. Capítulo.
  }

  \item Entrada na base de dados:

  \formato{
    @book\{rótulo,\\
      organization = \{B{\'i}blia\},\\
      type = \{A.T. ou N.T. título da parte referenciada. idioma\},\\
      title = \{título da obra\},\\
      edition = \{edição\},\\
      address = \{local de publicação\},\\
      publisher = \{editora\},\\
      year = \{data de publicação\},\\
      note = \{capítulo da parte referenciada\}\}\\
  }
\end{itemize}

\begin{itemize}[label={Ex.:},leftmargin=\parindent,parsep=0pt,itemsep=0pt]
  \item \formato{\citetext{bib:Biblia1982}}

  \begin{itemize}[leftmargin=*,parsep=0pt,itemsep=0pt]
    \item Entrada na base de dados:

    \formato{
      @book\{bib:Biblia1982,\\
      organization = \{B{\'i}blia\},\\
      type = \{A.T. G{\^e}nesis. Portugu{\^e}s\},\\
      title = \{B{\' \i}blia Sagrada\},\\
      edition = \{34\},\\
      address = \{S{\~a}o Paulo\},\\
      publisher = \{Ed. Ave Maria\},\\
      year = \{1982\},\\
      note = \{{c}ap. 19\}\}\\
    }

    \item Citação direta: \citeonline{bib:Biblia1982}
    \item Citação indireta: \cite{bib:Biblia1982}
  \end{itemize}
\end{itemize}

%~~~~~~~~~~~~~~~~~~~~~~~~~~~~~~~~~~~~~~~~~~~~~~~~~~~~~~
\subsubsection{Trabalhos apresentados em eventos (congressos, conferências, seminários etc.)}
%~~~~~~~~~~~~~~~~~~~~~~~~~~~~~~~~~~~~~~~~~~~~~~~~~~~~~~

\begin{itemize}[leftmargin=\parindent,parsep=0pt,itemsep=0pt]
  \item Formatação da referência:

  \formato{
    AUTOR(ES) DO TRABALHO. Título do trabalho. In: NOME DO EVENTO, número (se houver), 
    ano de realização, local de realização (cidade). \textsl{Título do documento}. 
    Local de publicação: Editora, data de publicação. Páginas inicial e final do trabalho.
  }

  \item Entrada na base de dados:

  \formato{
    @inproceedings\{rótulo,\\
      author = \{autor(es) do trabalho\},\\
      title = \{título do trabalho\},\\
      organization = \{nome do evento\},\\
      conference-number = \{número do evento\},\\
      conference-year = \{ano da realização\},\\
      conference-location = \{local de realização\},\\
      booktitle = \{título\},\\
      address = \{local de publicação\},\\
      publisher = \{editora\},\\
      year = \{data de publicação\},\\
      pages = \{página inicial -- página final\}\}\\
  }
\end{itemize}

\begin{itemize}[label={Ex.:},leftmargin=\parindent,parsep=0pt,itemsep=0pt]
  \item \formato{\citetext{bib:Machado1998}}

  \begin{itemize}[leftmargin=*,parsep=0pt,itemsep=0pt]
    \item Entrada na base de dados:

    \formato{
      @inproceedings\{bib:Machado1998,\\
        author = \{Machado, Caio G. and Rodrigues, N{\'\i}vea M. R.\},\\
        title = \{Altera{\c c}{\~a}o de altura de forrageamento de esp{\'e}cies 
                 de aves quando associadas a bandos mistos\},\\
        organization = \{Congresso Brasileiro de Ornitologia\},\\
        conference-number = \{7\},\\
        conference-year = \{1998\},\\
        conference-location = \{Rio de Janeiro\},\\
        booktitle = \{Resumos...\},\\
        address = \{Rio de Janeiro\},\\
        publisher = \{UERJ, NAPE\},\\
        year = \{1998\},\\
        pages = \{60-85\}\}\\
    }

    \item Citação direta: \citeonline{bib:Machado1998}
    \item Citação indireta: \cite{bib:Machado1998}
  \end{itemize}
\end{itemize}

%~~~~~~~~~~~~~~~~~~~~~~~~~~~~~~~~~~~~~~~~~~~~~~~~~~~~~~
\subsubsection{Volume específico, fascículo, suplemento, número especial de uma publicação pe\-ri\-ó\-di\-ca}
%~~~~~~~~~~~~~~~~~~~~~~~~~~~~~~~~~~~~~~~~~~~~~~~~~~~~~~

\begin{itemize}[leftmargin=\parindent,parsep=0pt,itemsep=0pt]
  \item Formatação da referência sem título próprio:

  \formato{
    TÍTULO DA PUBLICAÇÃO. Local: Editora, indicação de volume, número e 
    data (dia, mês e ano).
  }

  \item Entrada na base de dados:

  \formato{
    @journalpart\{rótulo,\\
      title = \{título da publicação\},\\
      address = \{local de publicação\},\\
      publisher = \{editora\},\\
      volume = \{volume\},\\
      number = \{número\},\\
      month = \{dia mês.\},\\
      year = \{ano de publicação\}\}\\
  }
\end{itemize}

\begin{itemize}[label={Ex.:},leftmargin=\parindent,parsep=0pt,itemsep=0pt]
  \item \formato{\citetext{bib:Sbpc2006}}

  \begin{itemize}[leftmargin=*,parsep=0pt,itemsep=0pt]
    \item Entrada na base de dados:

    \formato{
      @journalpart\{bib:Sbpc2006,\\
        title = \{Ci{\^e}ncia Hoje\},\\
        address = \{S{\~a}o Paulo\},\\
        publisher = \{SBPC\},\\
        volume = \{39\},\\
        number = \{229\},\\
        month = \{ago.\},\\
        year = \{2006\}\}\\
    }

    \item Citação direta: \citeonline{bib:Sbpc2006}
    \item Citação indireta: \cite{bib:Sbpc2006}\\
  \end{itemize}
\end{itemize}

\begin{itemize}[leftmargin=\parindent,parsep=0pt,itemsep=0pt]
  \item Formatação da referência com título próprio:

  \formato{
    TÍTULO DO FASCÍCULO. \textsl{Título da publicação}, Local de publicação, indicação 
    de volume, número, data (mês e ano) do fascículo. Nota indicativa do tipo de fascículo.
  }

  \item Entrada na base de dados:

  \formato{
    @article\{rótulo,\\
      title = \{título do fascículo\},\\
      journal = \{título da publicação\},\\
      address = \{local de publicação\},\\
      volume = \{indicação do volume\},\\
      number = \{indicação do número\},\\
      month = \{mês de publicação\},\\
      year = \{ano de publicação\},\\
      note = \{nota indicativa\}\}\\
  }
\end{itemize}

\begin{itemize}[label={Ex.:},leftmargin=\parindent,parsep=0pt,itemsep=0pt]
  \item \formato{\citetext{bib:Sesi2006}}

  \begin{itemize}[leftmargin=*,parsep=0pt,itemsep=0pt]
    \item Entrada na base de dados:

    \formato{
      @article\{bib:Sesi2006,\\
        title = \{Sesi 60 anos\},\\
        journal = \{Ind{\'u}stria Brasileira\},\\
        address = \{Bras{\'\i}lia\},\\
        volume = \{ano 6\},\\
        number = \{67 A\},\\
        month = \{set.\},\\
        year = \{2006\},\\
        note = \{Edi{\c c}{\~a}o especial\}\}\\
    }

    \item Citação direta: \citeonline{bib:Sesi2006}
    \item Citação indireta: \cite{bib:Sesi2006}
  \end{itemize}
\end{itemize}

%~~~~~~~~~~~~~~~~~~~~~~~~~~~~~~~~~~~~~~~~~~~~~~~~~~~~~~
\subsubsection{Artigos de periódicos (revistas, boletins etc.)}
%~~~~~~~~~~~~~~~~~~~~~~~~~~~~~~~~~~~~~~~~~~~~~~~~~~~~~~

\begin{itemize}[leftmargin=\parindent,parsep=0pt,itemsep=0pt]
  \item Formatação da referência:

  \formato{
    AUTOR(ES) DO ARTIGO. Título do artigo. \textsl{Título da revista}, local de 
    publicação, número do volume e/ou ano, número do fascículo, páginas inicial 
    e final do artigo, mês (abreviado) e ano do fascículo.
  }

  \item Entrada na base de dados:

  \formato{
    @article\{rótulo,\\
      author = \{autor(es) do artigo\},\\
      title = \{título do artigo\},\\
      journal = \{título da revista\},\\
      address = \{local de publicação\},\\
      volume = \{número do volume e/ou ano\},\\
      number = \{número do fascículo\},\\
      pages = \{página inicial -- página final\},\\
      month = \{mês\},\\
      year = \{ano do fascículo\}\}\\
  }
\end{itemize}

\begin{itemize}[label={Ex.:},leftmargin=\parindent,parsep=0pt,itemsep=0pt]
  \item \formato{\citetext{bib:Moura1983}}

  \begin{itemize}[leftmargin=*,parsep=0pt,itemsep=0pt]
    \item Entrada na base de dados:

    \formato{
      @article\{bib:Moura1983,\\
        author = \{Moura, Alexandrina Sobreira de\},\\
        title = \{Direito de habita{\c c}{\~a}o {\`a}s classes de baixa renda\},\\
        journal = \{Ci{\^e}ncia \& Tr{\'o}pico\},\\
        address = \{Recife\},\\
        volume = \{11\},\\
        number = \{1\},\\
        pages = \{71-78\},\\
        month = \{jan./jun.\},\\
        year = \{1983\}\}\\
    }

    \item Citação direta: \citeonline{bib:Moura1983}
    \item Citação indireta: \cite{bib:Moura1983}
  \end{itemize}
\end{itemize}

%~~~~~~~~~~~~~~~~~~~~~~~~~~~~~~~~~~~~~~~~~~~~~~~~~~~~~~
\subsubsection{Artigos de jornais}
%~~~~~~~~~~~~~~~~~~~~~~~~~~~~~~~~~~~~~~~~~~~~~~~~~~~~~~

\begin{itemize}[leftmargin=\parindent,parsep=0pt,itemsep=0pt]
  \item Formatação da referência:

  \formato{
    AUTOR(ES) DO ARTIGO. Título do artigo. \textsl{Título do jornal}, local de 
    publicação, data (dia, mês e ano). Título da seção, caderno ou parte, páginas 
    inicial e final do artigo.
  }

  \item Entrada na base de dados:

  \formato{
    @article\{rótulo,\\
      author = \{autor(es) do artigo\},\\
      title = \{titulo do artigo\},\\
      journal = \{título do jornal\},\\
      address = \{local de publicação\},\\
      month = \{dia mês da publicação\},\\
      year = \{ano da publicação\},\\
      note = \{título da seção, caderno ou parte, p. página inicial -- página final\}\}\\
  }

  \formato{
    @article\{rótulo,\\
      author = \{autor(es) do artigo\},\\
      title = \{titulo do artigo\},\\
      journal = \{título do jornal\},\\
      address = \{local de publicação\},\\
      pages = \{página inicial -- página final\},\\
      month = \{dia mês da publicação\},\\
      year = \{ano da publicação\}\}\\
  }
\end{itemize}

\begin{itemize}[label={Ex.:},leftmargin=\parindent,parsep=0pt,itemsep=0pt]
  \item \formato{\citetext{bib:Coutinho1985}}

  \begin{itemize}[leftmargin=*,parsep=0pt,itemsep=0pt]
    \item Entrada na base de dados:

    \formato{
      @article\{bib:Coutinho1985,\\
        author = \{Coutinho, Wilson\},\\
        title = \{O {P}a{\c c}o da {C}idade retorna ao seu brilho barroco\},\\
        journal = \{Jornal do Brasil\},\\
        address = \{Rio de Janeiro\},\\
        month = \{6 mar.\},\\
        year = \{1985\},\\
        note = \{Caderno B, p. 6\}\}\\
    }

    \item Citação direta: \citeonline{bib:Coutinho1985}
    \item Citação indireta: \cite{bib:Coutinho1985}\\
  \end{itemize}

  \item \formato{\citetext{bib:Cruvinel2006}}

  \begin{itemize}[leftmargin=*,parsep=0pt,itemsep=0pt]
    \item Entrada na base de dados:

    \formato{
      @article\{bib:Cruvinel2006,\\
        author = \{Cruvinel, Tereza\},\\
        title = \{Finan{\c c}as eleitorais\},\\
        journal = \{O Globo\},\\
        address = \{Rio de Janeiro\},\\
        pages = \{1\},\\
        month = \{29 nov.\},\\
        year = \{2006\}\\
    }

    \item Citação direta: \citeonline{bib:Cruvinel2006}
    \item Citação indireta: \cite{bib:Cruvinel2006}
  \end{itemize}
\end{itemize}

%~~~~~~~~~~~~~~~~~~~~~~~~~~~~~~~~~~~~~~~~~~~~~~~~~~~~~~
\subsubsection{Separatas}
%~~~~~~~~~~~~~~~~~~~~~~~~~~~~~~~~~~~~~~~~~~~~~~~~~~~~~~

\begin{itemize}[leftmargin=\parindent,parsep=0pt,itemsep=0pt]
  \item Formatação da referência de separatas de livros:

  \formato{
      AUTOR (da separata). Título (da separata). Local de publicação: Editora, 
      data de publicação. Separata de: AUTOR (da publicação principal). 
      \textsl{Título da publicação}. Local de publicação: Editora, data de 
      publicação. Paginação da separata.
  }

  \item Entrada na base de dados:

  \formato{
    @book\{rótulo,\\
      author = \{autor da separata\},\\
      title = \{título da separata\},\\
      address = \{local de publicação\},\\
      publisher = \{editora\},\\
      year = \{data de publicação\},\\
      note = \{Separata de: \textbackslash citetext\{rótulo da publicação principal\}\}\\
    }

  \formato{
    @book\{rótulo da publicação principal,\\
      editor = \{autor da publicação principal\},\\
      title = \{título da publicação\},\\
      address = \{local da publicação\},\\
      publisher = \{editora\},\\
      year = \{data da publicação\},\\
      note = \{paginação da separata\}\}\\
  }
\end{itemize}

\begin{itemize}[label={Ex.:},leftmargin=\parindent,parsep=0pt,itemsep=0pt]
  \item \formato{\citetext{bib:Knowles1961}}

%        \formato{\citetext{bib:Moore1960}}

  \begin{itemize}[leftmargin=*,parsep=0pt,itemsep=0pt]
    \item Entrada na base de dados:

    \formato{
      @book\{bib:Knowles1961,\\
        author = \{Knowles, William H.\},\\
        title = \{Industrial conflict and unions\},\\
        address = \{Berkeley\},\\
        publisher = \{Institute of Industrial Relations\},\\
        year = \{1961\},\\
        note = \{Separata de: \textbackslash citetext\{bib:Moore1960\}\}\\
    }

    \formato{
      @book\{bib:Moore1960,\\
        editor = \{Moore, Wilbert E.\},\\
        title = \{Labor commitment and social change in developing areas\},\\
        address = \{New York\},\\
        year = \{1960\},\\
        note = \{{p}. 291--312\}\}\\
    }

    \item Citação direta: \citeonline{bib:Knowles1961}
    \item Citação indireta: \cite{bib:Knowles1961}\\
  \end{itemize}
\end{itemize}

\begin{itemize}[leftmargin=\parindent,parsep=0pt,itemsep=0pt]
  \item Formatação da referência de separatas de periódicos:

  \formato{
    AUTOR (da separata). Título (da separata). Separata de: \textsl{Título do periódico}, 
    local de publicação, número do volume ou ano, número do fascículo, páginas inicial e 
    final da separata, data de publicação.
  }

  \item Entrada na base de dados:

  \formato{
      @article\{rótulo,\\
        author = \{autor da separata\},\\
        title = \{título da separata\},\\
        reprinted-text = \{Separata de\},\\
        journal = \{título do periódico\},\\
        address = \{local de publicação\},\\
        volume = \{número do volume\},\\
        number = \{número do fascículo\},\\
        pages = \{página inicial -- página final\},\\
        year = \{ano\}\}\\
    }
\end{itemize}

\begin{itemize}[label={Ex.:},leftmargin=\parindent,parsep=0pt,itemsep=0pt]
  \item \formato{\citetext{bib:Giacomel1989}}

  \begin{itemize}[leftmargin=*,parsep=0pt,itemsep=0pt]
    \item Entrada na base de dados:

    \formato{
      @article\{bib:Giacomel1989,\\
        author = \{Giacomel, F.\},\\
        title = \{Bionomia de {Hippopsis quinquelineata Aur. (Coleoptera, Cerambycidae)}\},\\
        reprinted-text = \{Separata de\},\\
        journal = \{Acta Biol{\'o}gica Paranaense\},\\
        volume = \{18\},\\
        number = \{1/4\},\\
        pages = \{63-72\},\\
        year = \{1989\}\}\\
    }

    \item Citação direta: \citeonline{bib:Giacomel1989}
    \item Citação indireta: \cite{bib:Giacomel1989}
  \end{itemize}
\end{itemize}

%::::::::::::::::::::::::::::::::::::::::::::::::::::::::::::::::::::::::::::
\subsection{Documentos em meio eletrônico}
%::::::::::::::::::::::::::::::::::::::::::::::::::::::::::::::::::::::::::::

Falta inserir texto introdutório...

%------------------------------------------------------------------
\subsubsection{Acesso online}
%------------------------------------------------------------------

\begin{itemize}[leftmargin=\parindent,parsep=0pt,itemsep=0pt]
  \item Formatação da referência:

  \formato{
    AUTOR(ES). Título do documento. Edição. Local de publicação: Editora, data de 
    publicação. Número de páginas ou volumes. Disponível em: $<$Endereço eletrônico$>$. 
    Acesso em: ... (data de acesso ao documento).
  }

  \item Entrada na base de dados:

  \formato{
    @book\{ rótulo,\\
      author = \{autor(es)\},\\
      title = \{título do documento\},\\
      edition = \{edição\},\\
      address = \{local de publicação\},\\
      publisher = \{editora\},\\
      year = \{data de publicação\},\\
      pages = \{número de páginas ou volumes\},\\
      url = \{endereço eletrônico\},\\
      urlaccessdate = \{data de acesso\}\}\\
  }

\end{itemize}

\begin{itemize}[label={Ex.:},leftmargin=\parindent,parsep=0pt,itemsep=0pt]
  \item \formato{\citetext{bib:Moura1996}}

  \begin{itemize}[leftmargin=*,parsep=0pt,itemsep=0pt]
    \item Entrada na base de dados:

    \formato{
      @book\{bib:Moura1996,\\
        author = \{Moura, Gevilacio Aguiar Coelho de\},\\
        title = \{Cita{\c c}{\~o}es e refer{\^e}ncias de documentos eletr{\^o}nicos\},\\
        edition = \{\},\\
        address = \{{[}S.l.\},\\
        publisher = \{s.n.\},\\
        year = \{19{---}{]}\},\\
        pages = \{86\},\\
        url = \{http://www.elogica.com.br/users/gmoura/reft\},\\
        urlaccessdate = \{9 dez. 1996\}\}\\
    }

    \item Citação direta: \citeonline{bib:Moura1996}
    \item Citação indireta: \cite{bib:Moura1996}
  \end{itemize}
\end{itemize}

%------------------------------------------------------------------
\subsubsection{FTP}
%------------------------------------------------------------------

\begin{itemize}[leftmargin=\parindent,parsep=0pt,itemsep=0pt]
  \item Formatação da referência:

  \formato{
    AUTOR (se conhecido). \textsl{Título}. Disponível em:$<$Endereço eletrônico$>$. 
    Acesso em: ... (data de acesso ao documento).
  }

  \item Entrada na base de dados:

  \formato{
      @misc\{rótulo,\\
        author = \{autor\},\\
        title = \{título\},\\
        url = \{endereço eletrônico\},\\
        urlaccessdate = \{data de acesso\}\}\\
  }

\end{itemize}

\begin{itemize}[label={Ex.:},leftmargin=\parindent,parsep=0pt,itemsep=0pt]
  \item \formato{\citetext{bib:Gates1996}}

  \begin{itemize}[leftmargin=*,parsep=0pt,itemsep=0pt]
    \item Entrada na base de dados:

    \formato{
      @misc\{bib:Gates1996,\\
        author = \{Gates, Garry\},\\
        title = \{Shakespeare and his Muse\},\\
        url = \{ftp://ftp.guten.net/bard/muse.txt\},\\
        urlaccessdate = \{1 out. 1996\}\}\\
    }

    \item Citação direta: \citeonline{bib:Gates1996}
    \item Citação indireta: \cite{bib:Gates1996}
  \end{itemize}
\end{itemize}

%------------------------------------------------------------------
\subsubsection{Lista de discussão}
%------------------------------------------------------------------

\begin{itemize}[leftmargin=\parindent,parsep=0pt,itemsep=0pt]
  \item Formatação da referência:

  \formato{
    TÍTULO da lista. Indicação de responsabilidade. Disponível em: $<$Endereço 
    eletrônico$>$. Acesso em: ... (data de acesso ao documento).
  }

  \item Entrada na base de dados:

  \formato{
    @misc\{rótulo,\\
      title = \{título da lista\},\\
      furtherresp = \{indicação de responsabilidade\},\\
      url = \{endereço eletrônico\},\\
      urlaccessdate = \{data de acesso\}\}\\
  }
\end{itemize}

\begin{itemize}[label={Ex.:},leftmargin=\parindent,parsep=0pt,itemsep=0pt]
  \item \formato{\citetext{bib:Ceatox2006}}

  \begin{itemize}[leftmargin=*,parsep=0pt,itemsep=0pt]
    \item Entrada na base de dados:

    \formato{
      @misc\{bib:Ceatox2006,\\
        title = \{Lista de discussão Ceatox\},\\
        furtherresp = \{Lista oferecida pela Faculdade de Inform{\'a}tica, 
                       Medicina e Setor de Toxicologia do Hospital Universit{\'a}rio 
                       Dr. Domingos Leonardo Ceravolo em conjunto com o Ceatox R 80 
                       de Presidente Prudente\},\\
        url = \{nettox-sbscribe@yahoogrupos.com.br\},\\
        urlaccessdate = \{13 set. 2006\}\}\\
    }

    \item Citação direta: \citeonline{bib:Ceatox2006}
    \item Citação indireta: \cite{bib:Ceatox2006}
  \end{itemize}
\end{itemize}

%------------------------------------------------------------------
\subsubsection{E-mail}
%------------------------------------------------------------------

\begin{itemize}[leftmargin=\parindent,parsep=0pt,itemsep=0pt]
  \item Formatação da referência:

  \formato{
    AUTOR DA MENSAGEM. \textsl{Título da mensagem} [mensagem pessoal]. Mensagem 
    recebida por $<$endereço eletrônico da pessoa que recebeu a mensagem$>$ em ... 
    (data do recebimento da mensagem).
  }

  \item Entrada na base de dados:

  \formato{
  }
\end{itemize}

\begin{itemize}[label={Ex.:},leftmargin=\parindent,parsep=0pt,itemsep=0pt]
  \item \formato{\citetext{}}

  \begin{itemize}[leftmargin=*,parsep=0pt,itemsep=0pt]
    \item Entrada na base de dados:

    \formato{
    }

    \item Citação direta: \citeonline{}
    \item Citação indireta: \cite{}
  \end{itemize}
\end{itemize}

%------------------------------------------------------------------
\subsubsection{Banco de dados}
%------------------------------------------------------------------

\begin{itemize}[leftmargin=\parindent,parsep=0pt,itemsep=0pt]
  \item Formatação da referência:

  \formato{
    NOME do Banco de dados. Disponível em: $<$Endereço eletrônico$>$. Acesso em: ...
    (data de acesso ao documento).
  }

  \item Entrada na base de dados:

  \formato{
      @misc\{rótulo,\\
        title = \{nome do banco de dados\},\\
        url = \{endereço eletrõnico\},\\
        urlaccessdate = \{data de acesso\}\}\\
  }
\end{itemize}

\begin{itemize}[label={Ex.:},leftmargin=\parindent,parsep=0pt,itemsep=0pt]
  \item \formato{\citetext{bib:Geodesicos2006}}

  \begin{itemize}[leftmargin=*,parsep=0pt,itemsep=0pt]
    \item Entrada na base de dados:

    \formato{
      @misc\{bib:Geodesicos2006,\\
        title = \{Banco de dados geod{\'e}sicos\},\\
        url = \{http://mapas.ibge.gov.br/geodesia2/viewer.htm\},\\
        urlaccessdate = \{23 set. 2006\}\}\\
    }

    \item Citação direta: \citeonline{bib:Geodesicos2006}
    \item Citação indireta: \cite{bib:Geodesicos2006}
  \end{itemize}
\end{itemize}

%------------------------------------------------------------------
\subsubsection{\textsl{Homepage} institucional}
%------------------------------------------------------------------

\begin{itemize}[leftmargin=\parindent,parsep=0pt,itemsep=0pt]
  \item Formatação da referência:

  \formato{
    TÍTULO DA HOMEPAGE. Indicações de responsabilidade (se houver). Descrição 
    sucinta do conteúdo da página. Disponível em: $<$Endereço eletrônico$>$. Acesso em: ...
    (data de acesso ao documento).
  }

  \item Entrada na base de dados:

  \formato{
      @misc\{rótulo,\\
        organization = \{título da homepage\},\\
        furtherresp = \{indicações de responsabilidade\},\\
        howpublished = \{descrição do conteúdo da página\},\\
        url = \{endereço eletrõnico\},\\
        urlaccessdate = \{data de acesso\}\}\\
  }

  \formato{
      @misc\{rótulo,\\
        title = \{título da homepage\},\\
        furtherresp = \{indicações de responsabilidade\},\\
        howpublished = \{descrição do conteúdo da página\},\\
        url = \{endereço eletrõnico\},\\
        urlaccessdate = \{data de acesso\}\}\\
  }
\end{itemize}

\begin{itemize}[label={Ex.:},leftmargin=\parindent,parsep=0pt,itemsep=0pt]
  \item \formato{\citetext{bib:Pinturabrasileira2005}}

  \begin{itemize}[leftmargin=*,parsep=0pt,itemsep=0pt]
    \item Entrada na base de dados:

    \formato{
      @misc\{bib:Pinturabrasileira2005,\\
        title = \{Arte e pintura brasileira: galeria virtual de arte\},\\
        howpublished = \{Apresenta reprodu{\c c}{\~o}es virtuais de pinturas brasileiras\},\\
        url = \{http://www.pinturabrasileira.com\},\\
        urlaccessdate = \{10 abr. 2005\}\}\\
    }

    \item Citação direta: \citeonline{bib:Pinturabrasileira2005}
    \item Citação indireta: \cite{bib:Pinturabrasileira2005}\\
  \end{itemize}

  \item \formato{\citetext{bib:ufjf2006}}

  \begin{itemize}[leftmargin=*,parsep=0pt,itemsep=0pt]
    \item Entrada na base de dados:

    \formato{
      @misc\{bib:ufjf2006,\\
        organization = \{Universidade Federal de Juiz de Fora\},\\
        furtherresp = \{Desenvolvido por {C}idaeli {I}nformática {L}tda\},\\
        howpublished = \{Apresenta informa{\c c}{\~o}es gerais sobre a universidade\},\\
        url = \{http://www.ufjf.br\},\\
        urlaccessdate = \{15 maio 2006\}\}\\
    }

    \item Citação direta: \citeonline{bib:ufjf2006}
    \item Citação indireta: \cite{bib:ufjf2006}
  \end{itemize}
\end{itemize}

%------------------------------------------------------------------
\subsubsection{Catálogo comercial em \textsl{homepage}}
%------------------------------------------------------------------

\begin{itemize}[leftmargin=\parindent,parsep=0pt,itemsep=0pt]
  \item Formatação da referência:

  \formato{
    TÍTULO DO CATÁLOGO. Indicação de responsabilidade (se houver). Disponível em: 
    $<$Endereço eletrônico$>$. Acesso em: ... (data de acesso ao documento).
  }

  \item Entrada na base de dados:

  \formato{
      @misc\{rótulo,\\
        title = \{nome do banco de dados\},\\
        furtherresp = \{indicação de responsabilidade\},\\
        url = \{endereço eletrõnico\},\\
        urlaccessdate = \{data de acesso\}\}\\
  }
\end{itemize}

\begin{itemize}[label={Ex.:},leftmargin=\parindent,parsep=0pt,itemsep=0pt]
  \item \formato{\citetext{bib:QualityMark2006}}

  \begin{itemize}[leftmargin=*,parsep=0pt,itemsep=0pt]
    \item Entrada na base de dados:

    \formato{
      @misc\{bib:QualityMark2006,\\
        title = \{Cat{\'a}logo {[}da{]} {Q}uality {M}ark {E}ditora\},\\
        url = \{http://www.qualitymark.com.br/catalog.aspx\},\\
        urlaccessdate = \{12 ago. 2006\}\}\\
    }

    \item Citação direta: \citeonline{bib:QualityMark2006}
    \item Citação indireta: \cite{bib:QualityMark2006}\\
  \end{itemize}

  \item \formato{\citetext{bib:Livrariasebo2006}}

  \begin{itemize}[leftmargin=*,parsep=0pt,itemsep=0pt]
    \item Entrada na base de dados:

    \formato{
      @misc\{bib:Livrariasebo2006,\\
        title = \{Livros usados: cat{\'a}logo\},\\
        url = \{http://livrariasebo.com.br/scripts/catalogo.asp?/ItemMenu=DiCom\},\\
        urlaccessdate = \{17 out. 2006\}\}\\
    }

    \item Citação direta: \citeonline{bib:Livrariasebo2006}
    \item Citação indireta: \cite{bib:Livrariasebo2006}
  \end{itemize}
\end{itemize}

%------------------------------------------------------------------
\subsubsection{Arquivo em disquete}
%------------------------------------------------------------------

\begin{itemize}[leftmargin=\parindent,parsep=0pt,itemsep=0pt]
  \item Formatação da referência:

  \formato{
    AUTOR(ES) DO ARQUIVO. \textsl{Nome do arquivo}. \textsl{extensão do arquivo}. 
    Título do documento (se houver). Local, data. Características físicas.
  }

  \item Entrada na base de dados:

  \formato{
      @manual\{rótulo,\\
        author = \{autor(es) do arquivo\},\\
        title = \{nome do arquivo.extensão do arquivo\},\\
        furtherresp = \{título do documento\},\\
        address = \{local\},\\
        month = \{dia mês.\},\\
        year = \{ano\},\\
        note = \{características físicas\}\}\\
  }
\end{itemize}

\begin{itemize}[label={Ex.:},leftmargin=\parindent,parsep=0pt,itemsep=0pt]
  \item \formato{\citetext{bib:Kraemer1995}}

  \begin{itemize}[leftmargin=*,parsep=0pt,itemsep=0pt]
    \item Entrada na base de dados:

    \formato{
      @manual\{bib:Kraemer1995,\\
        author = \{Kraemer, L{\'\i}gia Leindorf Bartz\},\\
        title = \{Apostila.doc\},\\
        address = \{Curitiba\},\\
        month = \{13 maio\},\\
        year = \{1995\},\\
        note = \{1 disquete, 3.5 pol. Word for Windows 6.0\}\}\\
    }

    \item Citação direta: \citeonline{bib:Kraemer1995}
    \item Citação indireta: \cite{bib:Kraemer1995}
  \end{itemize}
\end{itemize}

%------------------------------------------------------------------
\subsubsection{Base de dados}
%------------------------------------------------------------------

\begin{itemize}[leftmargin=\parindent,parsep=0pt,itemsep=0pt]
  \item Formatação da referência:

  \formato{
    AUTOR. \textsl{Título}. Local de publicação: Editora, data. Nome da base de dados, 
    versão (se houver).
  }

  \item Entrada na base de dados:

  \formato{
      @misc\{rótulo,\\
        organization = \{autor\},\\
        title = \{título\},\\
        address = \{local de publicação\},\\
        year = \{editora\},\\
        howpublished = \{nome da base de dados, versão\},\\
  }

  \formato{
      @misc\{rótulo,\\
        author = \{autor\},\\
        title = \{título\},\\
        address = \{local de publicação\},\\
        year = \{editora\},\\
        howpublished = \{nome da base de dados, versão\},\\
  }
\end{itemize}

\begin{itemize}[label={Ex.:},leftmargin=\parindent,parsep=0pt,itemsep=0pt]
  \item \formato{\citetext{bib:Viana2003}}

  \begin{itemize}[leftmargin=*,parsep=0pt,itemsep=0pt]
    \item Entrada na base de dados:

    \formato{
      @misc\{bib:Viana2003,\\
        organization = \{Biblioteca J. Baeta Viana\},\\
        title = \{Biblio\},\\
        address = \{Belo Horizonte\},\\
        year = \{2003\},\\
        howpublished = \{Base de dados em microisis\},\\
    }

    \item Citação direta: \citeonline{bib:Viana2003}
    \item Citação indireta: \cite{bib:Viana2003}
  \end{itemize}
\end{itemize}

%------------------------------------------------------------------
\subsubsection{CD-ROM}
%------------------------------------------------------------------

\begin{itemize}[leftmargin=\parindent,parsep=0pt,itemsep=0pt]
  \item Formatação da referência:

  \formato{
    AUTOR. \textsl{Título}: subtítulo (se houver). Local de publicação: Editora, data. 
    Tipo de suporte.
  }

  \item Entrada na base de dados:

  \formato{
      @book\{rótulo,\\
        author = \{autor\},\\
        title = \{título\},\\
        subtitle = \{subtítulo\},\\
        address = \{local\},\\
        publisher = \{editora\},\\
        year = \{data\},\\
        howpublished = \{tipo de suporte\}\}\\
  }

  \formato{
      @book\{rótulo,\\
        organization = \{organização\},\\
        type = \{informação complementar\},\\
        title = \{título\},\\
        subtitle = \{subtítulo\},\\
        address = \{local\},\\
        publisher = \{editora\},\\
        year = \{data\},\\
        howpublished = \{tipo de suporte\}\}\\
  }
\end{itemize}

\begin{itemize}[label={Ex.:},leftmargin=\parindent,parsep=0pt,itemsep=0pt]
  \item \formato{\citetext{bib:Winter1991}}

  \begin{itemize}[leftmargin=*,parsep=0pt,itemsep=0pt]
    \item Entrada na base de dados:

    \formato{
      @book\{bib:Winter1991,\\
        author = \{Winter, Robert\},\\
        title = \{Multimedia Stravinsky\},\\
        subtitle = \{an ilustrated, interactive musical exploration\},\\
        publisher = \{Microsoft\},\\
        year = \{c1991\},\\
        howpublished = \{1 CD-ROM\}\}\\
    }

    \item Citação direta: \citeonline{bib:Winter1991}
    \item Citação indireta: \cite{bib:Winter1991}\\
  \end{itemize}

  \item \formato{\citetext{bib:Biblia2002}}

  \begin{itemize}[leftmargin=*,parsep=0pt,itemsep=0pt]
    \item Entrada na base de dados:

    \formato{
      @book\{bib:Biblia2002,
        organization = \{B{\' \i}blia\},\\
        type = \{Portugu{\^e}s\},\\
        title = \{B{\' \i}blia Sagrada\},\\
        address = \{S{\~a}o Paulo\},\\
        publisher = \{Paulus\},\\
        year = \{2002\},\\
        howpublished = \{1 CD-ROM\}\}\\
    }

    \item Citação direta: \citeonline{bib:Biblia2002}
    \item Citação indireta: \cite{bib:Biblia2002}
  \end{itemize}
\end{itemize}

%======================================================================================
\section{Conclusão}
%======================================================================================

Inserir texto conclusivo...

% ----------------------------------------------------------
% Referencias
% ----------------------------------------------------------

\citeoption{abnt-options4}
\bibliography{abnt-options4,bibliografia,modelos_bibtex}

%---------------------------------------------------------------------
% INDICE REMISSIVO
%---------------------------------------------------------------------

%\printindex

\end{document}

